% RNN STATE CELL ====================================
\newcommand{\rnnstatecellU}[4]{

% operations
\node[state, minimum size=40pt,#4] (hprime#3) {$\textbf{h}^{#1}$};
\node[op, minimum size=40pt,below=15pt of hprime#3] (a#3) {$\textbf{a}^{#1}$};

\node[noop, below left=10pt and 10pt of a#3] (noop-left#3) {};
\node[state, minimum size=40pt, below left=12pt and 0.1pt of noop-left#3] (h#3) {$\textbf{h}^{#2}$};
\node[op, above left=8pt and 15pt of noop-left#3] (W#3) {W};

\node[noop, below right=10pt and 10pt of a#3] (noop-right#3) {};
\node[op, below right=12pt and 0.05pt of noop-right#3] (U#3) {U};

% outer operations
\node[op, minimum size=40pt, below=75pt of a#3] (x#3) {$\textbf{x}^{#1}$};

%% namedscope
\begin{scope}[on background layer]
\coordinate (p1#3) at (hprime#3.north);
\coordinate (p2#3) at (U#3.south east);
\coordinate (p3#3) at (h#3.south west);
\tkzCircumCenter(p1#3,p2#3,p3#3)
\tkzGetPoint{O}
\tkzDrawCircle[draw=orange, line width=1.5pt, fill=orange!60](O,p1#3)
\end{scope}

% edges
\path[tedge] (a#3) -- (hprime#3);
\path[tedge] (noop-left#3) -- (a#3);
\path[tedge] (noop-right#3) -- (a#3);
\path[tedge] (W#3) -- (noop-left#3);
\path[tedge] (h#3.north) to [bend left, out=70, distance=5pt] (noop-left#3.west);
\path[tedge] (U#3.north) to [bend right, out=-50, distance=5pt] (noop-right#3.east);
\path[tedge] (x#3) -- (noop-right#3);
}

% RNN OUTPUT ====================================
\newcommand{\rnnoutput}[3]{

% operations
\node[op, minimum size=40pt, #3] (o#2) {$\textbf{o}^{#1}$};
\node[op, minimum size=40pt, above=15pt of o#2] (yhat#2) {$\hat{\textbf{y}}^{#1}$};
\node[op, below left=1pt and 10pt of o#2] (V#2) {$\textbf{V}$};
\node[op, minimum size=40pt, left=50pt of yhat#2] (L#2) {$\textbf{L}^{#1}$};
\node[op, minimum size=40pt, left=20pt of L#2] (y#2) {$\textbf{y}^{#1}$};

% paths
\path[tedge] (o#2) -- (yhat#2);
\path[tedge] (yhat#2) -- (L#2);
\path[tedge] (y#2) -- (L#2);
\path[tedge] (V#2.north) to [bend left, out=90, in=135, distance=10pt] (o#2.west);

% namedscope
\begin{scope}[on background layer]
\coordinate (p1#2) at (yhat#2.north);
\coordinate (p2#2) at (V#2.south west);
\coordinate [right=35pt of o#2] (p3#2);
\tkzCircumCenter(p1#2,p2#2,p3#2)
\tkzGetPoint{O}
\tkzDrawCircle[draw=orange,line width=1.5pt, fill=orange!60](O,p1#2)
\end{scope}
}

\begin{figure}[H]
\hspace*{-1.0cm}
\scalebox{0.425}{
\begin{tikzpicture}[auto]

% timestep 1 cell
\rnnstatecellU{(1)}{(0)}{t1}{}

% timestep 2 cell
\rnnstatecellU{(2)}{(1)}{t2}{right=140pt of hprimet1};

% previous cell
\rnnstatecellU{(\tau-1)}{(\tau-2)}{prev}{right=200pt of hprimet2};

% next cell
\rnnstatecellU{(\tau)}{(\tau-1)}{next}{right=140pt of hprimeprev}

% state transfers
\path[tedge] (hprimet1.east) to [out=10, in=-160] (ht2.west);

\coordinate[right=30pt of hprimet2] (hppp);
\path[tedge] (hppp.east) to [out=10, in=-160] (hprev.west);

\path[tedge] (hprimeprev.east) to [out=10, in=-160] (hnext.west);

% ...
\node[fill=white, right=70pt of at2] (ppp) {\Huge{$\cdots$}};

% output
\rnnoutput{(\tau)}{output}{above=50pt of hprimenext};

\path[tedge] (hprimenext) -- (ooutput);

\end{tikzpicture}
}%\scalebox
\caption{The computational graph to compute the training loss of a RNN}
\end{figure}

