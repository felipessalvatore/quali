\documentclass[11pt,oneside,a4paper]{book}

% ---------------------------------------------------------------------------- %
% Pacotes 
\usepackage[T1]{fontenc}
% \usepackage[brazil]{babel}
\usepackage[latin1]{inputenc}
\usepackage[pdftex]{graphicx}           % usamos arquivos pdf/png como figuras
\usepackage{setspace}                   % espaçamento flexível
\usepackage{indentfirst}                % indentação do primeiro parágrafo
\usepackage{makeidx}                    % índice remissivo
\usepackage[nottoc]{tocbibind}          % acrescentamos a bibliografia/indice/conteudo no Table of Contents
\usepackage{courier}                    % usa o Adobe Courier no lugar de Computer Modern Typewriter
\usepackage{type1cm}                    % fontes realmente escaláveis
\usepackage{listings}                   % para formatar código-fonte (ex. em Java)
\usepackage{titletoc}
\usepackage{amsmath}
\usepackage{amssymb}
\usepackage{algorithm}
\usepackage[noend]{algpseudocode}
%\usepackage[bf,small,compact]{titlesec} % cabeçalhos dos títulos: menores e compactos
%\usepackage[fixlanguage]{babelbib}
\usepackage[font=small,format=plain,labelfont=bf,up,textfont=it,up]{caption}
\usepackage[usenames,svgnames,dvipsnames]{xcolor}
\usepackage{pgfplots}
\usepackage[a4paper,top=2.54cm,bottom=2.0cm,left=2.0cm,right=2.54cm]{geometry} % margens
%\usepackage[pdftex,plainpages=false,pdfpagelabels,pagebackref,colorlinks=true,citecolor=black,linkcolor=black,urlcolor=black,filecolor=black,bookmarksopen=true]{hyperref} % links em preto
\usepackage[pdftex,plainpages=false,pdfpagelabels,pagebackref,colorlinks=true,citecolor=DarkGreen,linkcolor=NavyBlue,urlcolor=DarkRed,filecolor=green,bookmarksopen=true]{hyperref} % links coloridos
\usepackage[all]{hypcap}                % soluciona o problema com o hyperref e capitulos
\usepackage[square,sort,nonamebreak,comma]{templates/natbib}  % citação bibliográfica alpha (alpha-ime.bst)
\fontsize{60}{62}\usefont{OT1}{cmr}{m}{n}{\selectfont}

% ---------------------------------------------------------------------------- %
% tikz
\usepackage{tikz}
\usetikzlibrary{calc,trees,positioning,arrows,chains}
%
\usepackage{tcolorbox}
%
\usepackage{pgfgantt}
%
\usepackage{todonotes}
%
\usepackage{enumitem}
\usepackage{color}
% ---------------------------------------------------------------------------- %

% ---------------------------------------------------------------------------- %
% Cabeçalhos similares ao TAOCP de Donald E. Knuth
\usepackage{fancyhdr}
\pagestyle{fancy}
\fancyhf{}
\renewcommand{\chaptermark}[1]{\markboth{\MakeUppercase{#1}}{}}
\renewcommand{\sectionmark}[1]{\markright{\MakeUppercase{#1}}{}}
\renewcommand{\headrulewidth}{0pt}

% ---------------------------------------------------------------------------- %
\graphicspath{{./figures/}}             % caminho das figuras (recomendável)
\frenchspacing                          % arruma o espaço: id est (i.e.) e exempli gratia (e.g.) 
\urlstyle{same}                         % URL com o mesmo estilo do texto e não mono-spaced
\makeindex                              % para o índice remissivo
\raggedbottom                           % para não permitir espaços extra no texto
\fontsize{60}{62}\usefont{OT1}{cmr}{m}{n}{\selectfont}
\cleardoublepage
\normalsize

% ---------------------------------------------------------------------------- %
% Opções de listing usados para o código fonte
% Ref: http://en.wikibooks.org/wiki/LaTeX/Packages/Listings
\lstset{ %
language=C,                  % choose the language of the code
basicstyle=\footnotesize,       % the size of the fonts that are used for the code
numbers=left,                   % where to put the line-numbers
numberstyle=\footnotesize,      % the size of the fonts that are used for the line-numbers
stepnumber=1,                   % the step between two line-numbers. If it's 1 each line will be numbered
numbersep=5pt,                  % how far the line-numbers are from the code
showspaces=false,               % show spaces adding particular underscores
showstringspaces=false,         % underline spaces within strings
showtabs=false,                 % show tabs within strings adding particular underscores
frame=single,                   % adds a frame around the code
framerule=0.6pt,
tabsize=4,                      % sets default tabsize to 2 spaces
captionpos=b,                   % sets the caption-position to bottom
breaklines=true,                % sets automatic line breaking
breakatwhitespace=false,        % sets if automatic breaks should only happen at whitespace
escapeinside={\%*}{*)},         % if you want to add a comment within your code
backgroundcolor=\color[rgb]{1.0,1.0,1.0}, % choose the background color.
rulecolor=\color[rgb]{0.8,0.8,0.8},
extendedchars=true,
xleftmargin=10pt,
xrightmargin=10pt,
framexleftmargin=10pt,
framexrightmargin=10pt
}

\definecolor{gray}{rgb}{0.4,0.4,0.4}
\definecolor{darkblue}{rgb}{0.0,0.0,0.6}
\definecolor{cyan}{rgb}{0.0,0.6,0.6}
\lstset{
  %basicstyle=\ttfamily,
  %columns=fullflexible,
  %showstringspaces=false,
  commentstyle=\color{gray}\upshape}
\lstdefinelanguage{XML}{
  morecomment=[s]{<!--}{-->},
  morecomment=[s]{<!-- }{ -->},
  morecomment=[n]{<!--}{-->},
  morecomment=[n]{<!-- }{ -->},
  morestring=[b]",
  morestring=[s]{>}{<},
  morecomment=[s]{<?}{?>},
  stringstyle=\color{black},
  identifierstyle=\color{darkblue},
  keywordstyle=\color{cyan},
  morekeywords={xmlns,version,type}% list your attributes here
}
\lstset{language=XML}

% ---------------------------------------------------------------------------- %
% Corpo do texto
\begin{document}
\frontmatter 
% cabeçalho para as páginas das seções anteriores ao capítulo 1 (frontmatter)
\fancyhead[RO]{{\footnotesize\rightmark}\hspace{2em}\thepage}
\setcounter{tocdepth}{2}
\fancyhead[LE]{\thepage\hspace{2em}\footnotesize{\leftmark}}
\fancyhead[RE,LO]{}
\fancyhead[RO]{{\footnotesize\rightmark}\hspace{2em}\thepage}

\onehalfspacing  % espaçamento

% ---------------------------------------------------------------------------- %
% CAPA
% Nota: O título para as dissertações/teses do IME-USP devem caber em um 
% orifício de 10,7cm de largura x 6,0cm de altura que há na capa fornecida pela SPG.
\thispagestyle{empty}
\begin{center}
    \vspace*{2.3cm}
    \textbf{\Large{Deep Active Learning for Sentiment Analysis}}\\
    
    \vspace*{1.2cm}
    \Large{Lucas Albuquerque Medeiros de Moura}
    
    \vskip 2cm
    \textsc{
    Text submitted\\[-0.25cm] 
    to\\[-0.25cm]
    Institute of Mathematics and Statistics\\[-0.25cm]
    of\\[-0.25cm]
    University of São Paulo\\[-0.25cm]
    for the\\[-0.25cm]
    Qualifying Examination for the Master Degree In\\[-0.25cm]
    Computer Science}
    
    \vskip 1.5cm
    %Programa: Computer Science\\
    Advisor: Professor Marcelo Finger

    \vskip 1cm
    \normalsize{This research is supported by Cnpq, Brazil}
    
    \vskip 0.5cm
    \normalsize{São Paulo, October, 2017}
\end{center}

% ---------------------------------------------------------------------------- %
% Página de rosto (SÓ PARA A VERSÃO DEPOSITADA - ANTES DA DEFESA)
% Resolução CoPGr 5890 (20/12/2010)
%
% IMPORTANTE:
%   Coloque um '%' em todas as linhas
%   desta página antes de compilar a versão
%   final, corrigida, do trabalho
%
%
%\newpage
%\thispagestyle{empty}
    %\begin{center}
        %\vspace*{2.3 cm}
        %\textbf{\Large{Título do trabalho a ser apresentado à \\
        %CPG para a dissertação/tese}}\\
        %\vspace*{2 cm}
    %\end{center}

    %\vskip 2cm

    %\begin{flushright}
    %Esta é a versão original da dissertação/tese elaborada pelo\\
    %candidato (Nome Completo do Aluno), tal como \\
    %submetida à Comissão Julgadora.
    %\end{flushright}

%\pagebreak


% ---------------------------------------------------------------------------- %
% Página de rosto (SÓ PARA A VERSÃO CORRIGIDA - APÓS DEFESA)
% Resolução CoPGr 5890 (20/12/2010)
%
% Nota: O título para as dissertações/teses do IME-USP devem caber em um 
% orifício de 10,7cm de largura x 6,0cm de altura que há na capa fornecida pela SPG.
%
% IMPORTANTE:
%   Coloque um '%' em todas as linhas desta
%   página antes de compilar a versão do trabalho que será entregue
%   à Comissão Julgadora antes da defesa
%
%
%\newpage
%\thispagestyle{empty}
    %\begin{center}
        %\vspace*{2.3 cm}
        %\textbf{\Large{Título do trabalho a ser apresentado à \\
        %CPG para a dissertação/tese}}\\
        %\vspace*{2 cm}
    %\end{center}

    %\vskip 2cm

    %\begin{flushright}
    %Esta versão da dissertação/tese contém as correções e alterações sugeridas\\
    %pela Comissão Julgadora durante a defesa da versão original do trabalho,\\
    %realizada em 14/12/2010. Uma cópia da versão original está disponível no\\
    %Instituto de Matemática e Estatística da Universidade de São Paulo.

    %\vskip 2cm

    %\end{flushright}
    %\vskip 4.2cm

    %\begin{quote}
    %\noindent Comissão Julgadora:
    
    %\begin{itemize}
        %\item Profª. Drª. Nome Completo (orientadora) - IME-USP [sem ponto final]
        %\item Prof. Dr. Nome Completo - IME-USP [sem ponto final]
        %\item Prof. Dr. Nome Completo - IMPA [sem ponto final]
    %\end{itemize}
      
    %\end{quote}
%\pagebreak


%\pagenumbering{roman}     % começamos a numerar 

% ---------------------------------------------------------------------------- %
% Agradecimentos:
% Se o candidato não quer fazer agradecimentos, deve simplesmente eliminar esta página 
%\chapter*{Agradecimentos}
%Texto texto texto texto texto texto texto texto texto texto texto texto texto
%texto texto texto texto texto texto texto texto texto texto texto texto texto
%texto texto texto texto texto texto texto texto texto texto texto texto texto
%texto texto texto texto. Texto opcional.


% ---------------------------------------------------------------------------- %
% Abstract
%\chapter*{Abstract}
%\noindent RIBEIRO, A. C. \textbf{Ranking warnings based on continuous source code static analysis}. 
%2010. 120 f.
%Tese (Doutorado) - Instituto de Matemática e Estatística,
%Universidade de São Paulo, São Paulo, 2010.
%\\

\input chapters/abstract.tex

% ---------------------------------------------------------------------------- %
% Sumário
\tableofcontents    % imprime o sumário

% ---------------------------------------------------------------------------- %
\chapter{Abbreviations}
\begin{tabular}{ll}
  SVM & Support Vector Machines\\
  KL & Kullback-Leibler divergence\\
  ELBO & Evidence Lower Bound\\
  LC & Least Confident\\
  H & Entropy\\
  I & Mutual Information\\
  LSTM & Long Short Term Memory\\
  ALU & Active Learning with Monte Carlo Dropout\\
  ALS & Active Learning with Softmax\\
\end{tabular}

% ---------------------------------------------------------------------------- %
%\chapter{Lista de Símbolos}
%\begin{tabular}{ll}
        %$\omega$    & Frequência angular\\
%\end{tabular}

% ---------------------------------------------------------------------------- %
% Listas de figuras e tabelas criadas automaticamente
%\listoffigures            
%\listoftables            

% ---------------------------------------------------------------------------- %
% Capítulos do trabalho
\mainmatter

% cabeçalho para as páginas de todos os capítulos
\fancyhead[RE,LO]{\thesection}

\singlespacing              % espaçamento simples
%\onehalfspacing            % espaçamento um e meio

\input {chapters/introduction.tex}
\input {chapters/related_work.tex}
\input {chapters/background.tex}
\input {chapters/experimental_design.tex}
%\input {chapters/team.tex}
\input {chapters/roadmap.tex}

% cabeçalho para os apêndices
\renewcommand{\chaptermark}[1]{\markboth{\MakeUppercase{\appendixname\ \thechapter}} {\MakeUppercase{#1}} }
\fancyhead[RE,LO]{}
\appendix

\clearpage
\input {appendices/elbo.tex}
\clearpage

% ---------------------------------------------------------------------------- %
% Bibliografia
\backmatter \singlespacing   % espaçamento simples
\bibliographystyle{templates/alpha-ime}% citação bibliográfica alpha
\bibliography{bibliografia}  % associado ao arquivo: 'bibliografia.bib'

% ---------------------------------------------------------------------------- %
% Índice remissivo
%\index{TBP|see{periodicidade região codificante}}
%\index{DSP|see{processamento digital de sinais}}
%\index{STFT|see{transformada de Fourier de tempo reduzido}}
%\index{DFT|see{transformada discreta de Fourier}}
%\index{Fourier!transformada|see{transformada de Fourier}}

%\printindex   % imprime o índice remissivo no documento 

\end{document}