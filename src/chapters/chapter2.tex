\chapter{Generative models for text}\label{gen}

Different deep learning architectures are used in NLP: \textbf{convolutional architectures} have a good performance in tasks were it is required to find a linguistic indicator regardless of its position (e.g., document classification, short-text categorization, sentiment classification, etc); high quality word embeddings can be achieved with models that are a kind of \textbf{feedforward neural network} \cite{Mikolov23}. But for a variety of works in natural language we want to capture regularities and similarities in a text structure. That is way \textbf{recurrent} and \textbf{recursive} models have been widely used in the field. Here we are focused on generative models and since recurrent models have been producing very strong results for language modeling \cite{goldberg15}, we will concentrate on them.

\section{RNN}

Recurrent Neural Network is a family of neural network specialized in sequential data $\vect{x}_1, \dots, \vect{x}_\tau$. As a neural network, a RNN is a parametrized function that we use to approximate one hidden function from the data. As before we can take the simplest RNN as a neural network with only one hidden layer. But now, what make RNNs unique is a recurrent definition of one of its hidden layer:

\begin{equation}
\vect{h}^{(t)} = g(\vect{h}^{(t-1)}, \vect{x}^{(t)}; \vect{\theta})
\end{equation}

$\vect{h}^{(t)}$ is called \textit{state}, \textit{hidden state}, or \textbf{cell}.

Is costumerely to represent a RNN as a ciclic graph \ref{RNNSimplified}

\begin{figure}[H]
\label{RNNSimplified}
\centering

\scalebox{1.40}{
\begin{tikzpicture}[auto]

% RNN state cell =============================
\node[state] (h) {$\vect{h}$};
\node[op, below=30pt of h] (x) {$\vect{x}$};
\node[op, above=30pt of h] (yhat) {$\hat{\vect{y}}$};



% edges
\path[tedge] (x) edge node[below right= -4pt] {}  (h) ;
\path[tedge] (h) edge [out=-400,in=-320,looseness=8, distance=125pt] node[above right] {} (h);
\path[tedge] (h) edge node[below right = -4pt] {} (yhat);


\end{tikzpicture}
} % scalebox
\caption{Ciclic representation}
\end{figure}


\par This recurrent equation can be unfolded for a finite number of steps $\tau$. For example, when $\tau =3$:
\vspace{0.2cm}

\begin{align}
\vect{h}^{(3)}& = g(\vect{h}^{(2)}, \vect{x}^{(3)}; \vect{\theta})\\
 & = g(g(\vect{h}^{(1)}, \vect{x}^{(2)}; \vect{\theta}), \vect{x}^{(3)}; \vect{\theta})\\
 & = g(g(g(\vect{h}^{(0)}, \vect{x}^{(1)}; \vect{\theta}), \vect{x}^{(2)}; \vect{\theta}), \vect{x}^{(3)}; \vect{\theta})\\
\end{align}



Hence for any finite step $\tau$ we can describe the model as a DAG \ref{RNNSimplifiedUnfolded}.

% RNN STATE CELL ====================================
\newcommand{\rnnSimpleUU}[4]{

% operations
\node[state, minimum size=40pt,#4] (h#3) {$\vect{h}^{#1}$};
\node[op, minimum size=40pt,below=30pt of h#3] (x#3) {$\vect{x}^{#1}$};
\node[op, minimum size=40pt, above=30pt of h#3] (yhat#3) {$\hat{\vect{y}}^{#1}$};

% edges
\path[tedge] (x#3) edge node[below right= -4pt] {} (h#3);
\path[tedge] (h#3) edge node[below right = -4pt] {} (yhat#3);
}

\begin{figure}[H]
\label{RNNSimplifiedUnfolded}
\hspace*{-1.0cm}
\scalebox{0.9}{
\begin{tikzpicture}[auto]

% timestep 1
\rnnSimpleUU{(1)}{(0)}{t1}{}

% % timestep 0
\node[state, minimum size=40pt,left=50pt of ht1] (ht0) {$\vect{h}^{(0)}$};

% % timestep 2
\rnnSimpleUU{(2)}{(1)}{t2}{right=50pt of ht1};


% % timestep 2
\rnnSimpleUU{(3)}{(1)}{t3}{right=50pt of ht2};


% % state transfers
\path[tedge] (ht0) edge node[above right = 2pt] {} (ht1);
\path[tedge] (ht1) edge node[above right = 2pt] {} (ht2);
\path[tedge] (ht2) edge node[above right = 2pt] {} (ht3);

\end{tikzpicture}
}%\scalebox
\caption{Unfolded computational graph}
\end{figure}



Using a concret example consider the following model define by the equations:

\begin{equation}
f(\vect{x}^{(t)}, \vect{h}^{(t-1)}; \vect{V}, \vect{W}, \vect{U}, \vect{c}, \vect{b}) = \vect{\hat{y}}^{(t)}
\end{equation}
 \vspace{0.2cm}
\begin{equation}
\vect{\hat{y}}^{(t)} = softmax(\vect{V} \vect{h}^{(t)} + \vect{c})
\end{equation}
\vspace{0.2cm}
 \begin{equation}
\vect{h}^{(t)} = g(\vect{h}^{(t-1)}, \vect{x}^{(t)}; \vect{W},\vect{U}, \vect{b})
\end{equation}
\vspace{0.2cm}
\begin{equation}
\vect{h}^{(t)} = \sigma(\vect{W} \vect{h}^{(t-1)} + \vect{U} \vect{x}^{(t)} + \vect{b})
\end{equation}

Using the graphical representation the model can be view as:

\begin{figure}[H]
\label{RNNGraphExpanded}
\centering

\scalebox{0.75}{
\begin{tikzpicture}[auto]

% RNN state cell =============================
\node[state] (hprime) {$\vect{h}^{\prime}$};
\node[op, below=15pt of hprime] (a) {$\vect{a}$};

\node[op, right=25pt of a] (b) {$\vect{b}$};
\node[noop, below left=10pt and 10pt of a] (noop-left) {};
\node[state, below left=12pt and 0.1pt of noop-left] (h) {$\vect{h}$};
\node[op, above left=8pt and 15pt of noop-left] (W) {$\vect{W}$};

\node[noop, below right=10pt and 10pt of a] (noop-right) {};
\node[op, below right=12pt and 0.05pt of noop-right] (U) {$\vect{U}$};


% outer operations
\node[op, below=75pt of a] (x) {$\vect{x}$};

%% namedscope
\begin{scope}[on background layer]
\coordinate (p1) at (hprime.north);
\coordinate (p2) at (U.south east);
\coordinate (p3) at (h.south west);
\tkzCircumCenter(p1,p2,p3)
\tkzGetPoint{O}
\tkzDrawCircle[draw=orange, line width=1.5pt, fill=orange!60](O,p1)
\end{scope}

% edges

\path[tedge] (a) edge node[below right= -7pt] {$\;\; \Large{\sigma}$}  (hprime);
\path[tedge] (b) -- (a);
\path[tedge] (noop-left) -- (a);
\path[tedge] (noop-right) -- (a);
\path[tedge] (W) -- (noop-left);
\path[tedge] (h.north) to [bend left, out=70, distance=5pt] (noop-left.west);
\path[tedge] (U.north) to [bend right, out=-50, distance=5pt] (noop-right.east);
\path[tedge] (x) -- (noop-right);

% RNN output cell =============================

% operations
\node[noop, above=25pt of hprime] (noop-center) {};
\node[op, above=50pt of hprime] (o) {$\vect{o}$};
\node[op, right=15pt of o] (c) {$\vect{c}$};
\node[op, above=15pt of o] (yhat) {$\hat{\textbf{y}}$};
\node[op, below left=1pt and 10pt of o] (V) {$\vect{V}$};

% paths
\path[tedge] (hprime) -- (noop-center);
\path[tedge] (o) edge node[below right= -10pt] {$\;\;$softmax} (yhat);
\path[tedge] (c) -- (o);
\path[tedge] (V) -- (noop-center);
\path[tedge] (noop-center) -- (o);

% namedscope
\begin{scope}[on background layer]
\coordinate (p4) at (yhat.north);
\coordinate (p5) at (V.south west);
\coordinate (p6) at (c.east);
\tkzCircumCenter(p4,p5,p6)
\tkzGetPoint{O}
\tkzDrawCircle[draw=orange, line width=1.5pt, fill=orange!60](O,p4)
\end{scope}

\end{tikzpicture}
} % scalebox
\caption{Graph of a RNN}
\end{figure}
 

!!! explicar como funciona o treinamento !!!

An RNN with a loss function:

% RNN STATE CELL ====================================
\newcommand{\rnnstatecellU}[4]{

% operations
\node[state, minimum size=40pt,#4] (hprime#3) {$\textbf{h}^{#1}$};
\node[op, minimum size=40pt,below=15pt of hprime#3] (a#3) {$\textbf{a}^{#1}$};

\node[noop, below left=10pt and 10pt of a#3] (noop-left#3) {};
\node[state, minimum size=40pt, below left=12pt and 0.1pt of noop-left#3] (h#3) {$\textbf{h}^{#2}$};
\node[op, above left=8pt and 15pt of noop-left#3] (W#3) {W};

\node[noop, below right=10pt and 10pt of a#3] (noop-right#3) {};
\node[op, below right=12pt and 0.05pt of noop-right#3] (U#3) {U};

% outer operations
\node[op, minimum size=40pt, below=75pt of a#3] (x#3) {$\textbf{x}^{#1}$};

%% namedscope
\begin{scope}[on background layer]
\coordinate (p1#3) at (hprime#3.north);
\coordinate (p2#3) at (U#3.south east);
\coordinate (p3#3) at (h#3.south west);
\tkzCircumCenter(p1#3,p2#3,p3#3)
\tkzGetPoint{O}
\tkzDrawCircle[draw=orange, line width=1.5pt, fill=orange!60](O,p1#3)
\end{scope}

% edges
\path[tedge] (a#3) -- (hprime#3);
\path[tedge] (noop-left#3) -- (a#3);
\path[tedge] (noop-right#3) -- (a#3);
\path[tedge] (W#3) -- (noop-left#3);
\path[tedge] (h#3.north) to [bend left, out=70, distance=5pt] (noop-left#3.west);
\path[tedge] (U#3.north) to [bend right, out=-50, distance=5pt] (noop-right#3.east);
\path[tedge] (x#3) -- (noop-right#3);
}

% RNN OUTPUT ====================================
\newcommand{\rnnoutput}[3]{

% operations
\node[op, minimum size=40pt, #3] (o#2) {$\textbf{o}^{#1}$};
\node[op, minimum size=40pt, above=15pt of o#2] (yhat#2) {$\hat{\textbf{y}}^{#1}$};
\node[op, below left=1pt and 10pt of o#2] (V#2) {$\textbf{V}$};
\node[op, minimum size=40pt, left=50pt of yhat#2] (L#2) {$\textbf{L}^{#1}$};
\node[op, minimum size=40pt, left=20pt of L#2] (y#2) {$\textbf{y}^{#1}$};

% paths
\path[tedge] (o#2) -- (yhat#2);
\path[tedge] (yhat#2) -- (L#2);
\path[tedge] (y#2) -- (L#2);
\path[tedge] (V#2.north) to [bend left, out=90, in=135, distance=10pt] (o#2.west);

% namedscope
\begin{scope}[on background layer]
\coordinate (p1#2) at (yhat#2.north);
\coordinate (p2#2) at (V#2.south west);
\coordinate [right=35pt of o#2] (p3#2);
\tkzCircumCenter(p1#2,p2#2,p3#2)
\tkzGetPoint{O}
\tkzDrawCircle[draw=orange,line width=1.5pt, fill=orange!60](O,p1#2)
\end{scope}
}

\begin{figure}[H]
\hspace*{-1.0cm}
\scalebox{0.425}{
\begin{tikzpicture}[auto]

% timestep 1 cell
\rnnstatecellU{(1)}{(0)}{t1}{}

% timestep 2 cell
\rnnstatecellU{(2)}{(1)}{t2}{right=140pt of hprimet1};

% previous cell
\rnnstatecellU{(\tau-1)}{(\tau-2)}{prev}{right=200pt of hprimet2};

% next cell
\rnnstatecellU{(\tau)}{(\tau-1)}{next}{right=140pt of hprimeprev}

% state transfers
\path[tedge] (hprimet1.east) to [out=10, in=-160] (ht2.west);

\coordinate[right=30pt of hprimet2] (hppp);
\path[tedge] (hppp.east) to [out=10, in=-160] (hprev.west);

\path[tedge] (hprimeprev.east) to [out=10, in=-160] (hnext.west);

% ...
\node[fill=white, right=70pt of at2] (ppp) {\Huge{$\cdots$}};

% output
\rnnoutput{(\tau)}{output}{above=50pt of hprimenext};

\path[tedge] (hprimenext) -- (ooutput);

\end{tikzpicture}
}%\scalebox
\caption{The computational graph to compute the training loss of a RNN}
\end{figure}




\section{GRU}


To capture long-term dependencies on a RNN  the authors of the paper \cite{ChungGCB14}  proposed a new architecture called \textbf{gated recurrent unit} (\textbf{GRU}). This model was constructed to make each hidden state  $\vect{h}^{(t)}$ to adaptively capture dependencies of different time steps. It work as follows, at each step $t$ one candidate for hidden state is formed:

\begin{equation}
\vect{\widetilde{h}}^{(t)} = tahn(\vect{W} (\vect{h}^{(t-1)} \odot  \vect{r}^{(t)}) + \vect{U} \vect{x}^{(t)} + \vect{b})
\end{equation}


where $\vect{r}^{(t)}$ is a vector with values in $[0, 1]$ called a \textbf{reset gate}, i.e.,  a vector that at each entry outputs the probability of reseting the  corresponding entry in the previous hidden state $\vect{h}^{(t-1)}$. Together with $\vect{r}^{(t)}$ we define an \textbf{update gate}, $\vect{u}^{(t)}$. It is also a vector with values in $[0, 1]$. Intuitively we can say that this vector decides how much on each dimension we will use the candidate update. Both $\vect{r}^{(t)}$ and $\vect{u}^{(t)}$ are defined by $\vect{h}^{(t-1)}$ and $\vect{x}^{(t)}$; they also have specific parameters:

\begin{equation}
\vect{r}^{(t)} = \sigma(\vect{W}_{r} \vect{h}^{(t-1)} + \vect{U}_{r} \vect{x}^{(t)} + \vect{b}_{r})
\end{equation}


\begin{equation}
\vect{u}^{(t)} = \sigma(\vect{W}_{u} \vect{h}^{(t-1)} + \vect{U}_{u} \vect{x}^{(t)} + \vect{b}_{u})
\end{equation}

At the end the new hidden state $\vect{h}^{(t)}$ is defined by the recurrence:

\begin{equation}
\vect{h}^{(t)} = \vect{u}^{(t)} \odot \vect{\widetilde{h}}^{(t)} + (1 - \vect{u}^{(t)}) \odot \vect{h}^{(t-1)} 
\end{equation}

Note that the new hidden state combines the candidate hidden state $\vect{\widetilde{h}}^{(t)}$ with the past hidden state $\vect{h}^{(t-1)}$ using both $\vect{r}^{(t)}$ and $\vect{u}^{(t)}$ to adaptively copy and forget information.

\par It can appear more complex, but we can view the GRU model just as a refinement of the standard RNN with a new computation for the hidden state. Let $\vect{\theta} = [ \vect{W},\vect{U},\vect{b} ]$, $\vect{\theta}_{u} = [ \vect{W}_{u},\vect{U}_{u},\vect{b}_{u} ]$  and $\vect{\theta}_{r} = [ \vect{W}_{r},\vect{U}_{r},\vect{b}_{r} ]$; and $\text{aff}(\vect{\theta})$ be the following operation:

\begin{equation}
\text{aff}(\vect{\theta}) = \vect{W} \vect{h} + \vect{U} \vect{x} + \vect{b}
\end{equation}

With similar definitions for $\text{aff}(\vect{\theta}_{u})$ and $\text{aff}(\vect{\theta}_{r})$. Figure \ref{GRU} shows the hidden state of the GRU model for time step $t$. If compared with Figure \ref{RNNGraphExpanded} we can see that the basic structure is the same, just the way of computing the hidden state $\vect{h}^{(t)}$ has changed.

 \begin{figure}[H]
\label{GRU}
\centering

\scalebox{0.45}{
\begin{tikzpicture}[auto]

% GRU state cell =============================
\node[state] (h) {$\vect{h}^{(t-1)}$};
\node[op, below right=25pt and 50pt of h] (affu) {$\text{aff}(\vect{\theta}_{u})$};
\node[op,right=95pt of affu] (affr) {$\text{aff}(\vect{\theta}_{r})$};
\node[op,above right= 30pt and 30pt of affr] (r) {$\vect{r}^{(t)}$};
\node[op,above left=25pt and 10pt of r] (hada1) {$\odot$};
\node[op,left=35pt of hada1] (aff) {$\text{aff}(\vect{\theta})$};
\node[op,above=25pt of aff] (htilde) {$\vect{\widetilde{h}}^{(t)}$};
\node[op,above=25pt of htilde] (hada2) {$\odot$};
\node[op,above=40pt of affu] (u) {$\vect{u}^{(t)}$};
\node[op,above=35pt of u] (1minusu) {$1 - \vect{u}^{(t)}$};
\node[op,above left =35pt and 35pt of 1minusu] (hada3) {$\odot$};
\node[state,above right=25pt and 30pt of hada2] (hprime) {$\vect{h}^{(t)}$};
\node[textonly,below right=15pt and 95pt of affu] (inv1) {};
\node[op,below left=40pt and 40pt of inv1] (x) {$\vect{x}^{(t)}$};
\node[textonly, left=35pt of h] (inv2) {};
\node[textonly, right=35pt of hprime] (inv3) {};
\node[textonly, above=35pt of hprime] (inv4) {};
\node[textonly, below=2pt of hprime] (name1) {$+$};

% namedscope
\begin{scope}[on background layer]
\coordinate (p1) at (hprime.north);
\coordinate (p2) at (inv1.south);
\coordinate (p3) at (h.west);
\tkzCircumCenter(p1,p2,p3)
\tkzGetPoint{O}
\tkzDrawCircle[draw=orange, line width=1.5pt, fill=orange!60](O,p1)
\end{scope}

% % edges
\path[tedge] (x) to [out=90, in=-90] (affu);
\path[tedge] (x)to [out=90, in=-90](affr);
\path[tedge] (x) to [out=90, in=-90](aff);
\path[tedge] (h) to [out=0] (affu);
\path[tedge] (h) to (hada3);
\path[tedge] (h) to [out=0, in=180] (affr);
\path[tedge] (h) to [out=0, in=-90] (hada1);
\path[tedge] (affu)edge node[above right= 7pt and -5pt] {$\;\; \Large{\sigma}$} (u);
\path[tedge] (affr) edge node[above right= - 7pt and -5pt] {$\;\; \Large{\sigma}$} (r);
\path[tedge] (r)--(hada1);
\path[tedge] (hada1)--(aff);
\path[tedge] (aff)edge node[above right= - 7pt and -5pt] {$\;\; \Large{tanh}$}(htilde);
\path[tedge] (htilde)--(hada2);
\path[tedge] (hada2)--(hprime);
\path[tedge] (hada3) to [out=70, in=150] (hprime);
\path[tedge] (u)--(1minusu);
\path[tedge] (u)--(hada2);
\path[tedge] (1minusu)--(hada3);
\path[tedge] (inv2)--(h);
\path[tedge] (hprime)--(inv3);
\path[tedge] (hprime)--(inv4);

\end{tikzpicture}
} % scalebox
\end{figure}


\section{LSTM}

% \begin{figure}[H]
\label{LSTM_cell}
\centering

\scalebox{0.75}{
\begin{tikzpicture}[auto]

% RNN state cell =============================
\node[state] (h) {$\vect{h}$};
\node[op, above right=25pt and 10pt of h] (add1) {$add(\vect{W}_{f},\vect{U}_{f})$};
\node[op,right=25pt of add1] (add2) {$add(\vect{W}_{i},\vect{U}_{i})$};
\node[op,right=25pt of add2] (add3) {$add(\vect{W}_{\tilde{c}},\vect{U}_{\tilde{c}})$};
\node[op,right=25pt of add3] (add4) {$add(\vect{W}_{o},\vect{U}_{o})$};
\node[op,above=20pt of add1] (f) {$\vect{f}$};
\node[op,above=20pt of add2] (i) {$\vect{i}$};
\node[op,above=20pt of add3] (tildec) {$\vect{\tilde{c}}$};
\node[op,above=20pt of add4] (o) {$\vect{o}$};
\node[op,above=20pt of f] (haddamar1) {$\circ$};
\node[state,above left=20pt and 20pt of haddamar1] (C) {$\vect{C}$};
\node[op,above right=20pt and 20pt of i] (haddamar2) {$\circ$};
\node[state,above left=20pt and 20pt of haddamar2] (Cprime) {$\vect{C}^{\prime}$};
\node[op, below right=3pt and 70pt of Cprime] (tanh) {$tanh$};
\node[state, right=50pt of o] (hprime) {$\vect{h}^{\prime}$};
\node[textonly,right=180pt of h] (inv1) {};
\node[op, below=30pt of inv1] (x) {$\vect{x}$};
\node[textonly, below=2pt of Cprime] (name1) {$+$};
\node[textonly, below=2pt of hprime] (name2) {$\circ$};




%% namedscope
\node[draw=orange, fit= (C) (h) (hprime), line width=1.5pt, fill=orange!60, fill opacity=0.2] {};




% edges
\path[tedge] (x) to (add1);
\path[tedge] (x)--(add2);
\path[tedge] (x)--(add3);
\path[tedge] (x)--(add4);
\path[tedge] (h)--(add1);
\path[tedge] (h) to [in=280, distance=5pt] (add2);
\path[tedge] (h)to [in=280, distance=5pt](add3);
\path[tedge] (h)to [in=280, distance=5pt](add4);
\path[tedge] (add1) edge node[below right= -7pt] {$\;\; \Large{\sigma}$}  (f);
\path[tedge] (add2) edge node[below right= -7pt] {$\;\; \Large{\sigma}$}  (i);
\path[tedge] (add3) edge node[below right= -7pt] {$\;\; \Large{tanh}$}  (tildec);
\path[tedge] (add4) edge node[below right= -7pt] {$\;\; \Large{\sigma}$}  (o);
\path[tedge] (f)--(haddamar1);
\path[tedge] (haddamar1)--(Cprime);
\path[tedge] (C)--(haddamar1);
\path[tedge] (i)--(haddamar2);
\path[tedge] (tildec)--(haddamar2);
\path[tedge] (Cprime)--(tanh);
\path[tedge] (haddamar2)--(Cprime);
\path[tedge] (tanh)--(hprime);
\path[tedge] (tanh)--(hprime);
\path[tedge] (o)--(hprime);

\end{tikzpicture}
} % scalebox
\caption{LSTM hidden cell}
\end{figure}



\section{Language model}

We call \textit{language model} a probability distribution over sequences of tokens in a natural language.

\[
P(x_1,x_2,x_3,x_4) = p
\]

This model is used for different nlp tasks such as speech recognition, machine translation, text auto-completion, spell correction, question answering, summarization and many others.

The classical approuch to a languange model was to use the chain rule and a markovian assumptiom, i.e., for a specific $n$ we assume that:

\begin{equation}
P(x_1, \dots, x_T) = \prod_{t=1}^{T} P(x_t \vert x_1, \dots, x_{t-1}) = \prod_{t=1}^{T} P(x_{t} \vert x_{t - (n+1)}, \dots, x_{t-1})
\end{equation} 


This gave raise to models based on $n$-gram statistics. The choice of $n$ yields different models; for example 
\textit{Unigram} language model ($n=1$): 
\begin{equation}
P_{uni}(x_1, x_2, x_3, x_4) = P(x_1)P(x_2)P(x_3)P(x_4)
\end{equation}

where $P(x_i) = count(x_i)$.\\

\textit{Bigram} language model ($n=2$): 
\begin{equation}
P_{bi}(x_1,x_2,x_3,x_4) = P(x_1)P(x_2\vert x_1)P(x_3\vert x_2)P(x_4\vert x_3)
\end{equation} 
where
\[
P(x_i\vert x_j) = \frac{count(x_i, x_j)}{count(x_j)}
\]


Higher $n$-grams yields better performance. But at the same time higher $n$-grams requires a lot of memory\cite{Heafield}.

Since \cite{Mikolov11} the approach has change, instead of using one approach that is specific for the language domain, we can use a general model for sequential data prediction: a RNN.

So, our learning task is to estimate the probability distribution 

\[
P(x_{n} = \text{word}_{j^{*}} | x_{1}, \dots ,x_{n-1})
\]

for any $(n-1)$-sequence of words $x_{1}, \dots ,x_{n-1}$.

We start with a corpus $C$ with $T$ tokens and a vocabulary $\Vocab$.\\\

Example: \textbf{Make Some Noise} by the Beastie Boys.\\

\begin{quote}
Yes, here we go again, give you more, nothing lesser\\
Back on the mic is the anti-depressor\\
Ad-Rock, the pressure, yes, we need this\\
The best is yet to come, and yes, believe this\\
... \\
\end{quote}

\begin{itemize}
\item $T = 378$
\item $|\Vocab| = 186$
\end{itemize}


The dataset is a collection of pairs $(\vect{x},\vect{y})$ where $\vect{x}$ is one word and $\vect{y}$ is the immediately next word. For example:
\begin{itemize}
\item [] $(\vect{x}^{(1)}, \vect{y}^{(1)}) =$ (Yes, here).
\item [] $(\vect{x}^{(2)}, \vect{y}^{(2)}) =$ (here, we)
\item [] $(\vect{x}^{(3)}, \vect{y}^{(3)}) =$ (we, go)
\item [] $(\vect{x}^{(4)}, \vect{y}^{(4)}) =$ (go, again)
\item [] $(\vect{x}^{(5)}, \vect{y}^{(5)}) =$ (again, give)
\item [] $(\vect{x}^{(6)}, \vect{y}^{(6)}) =$ (give, you)
\item [] $(\vect{x}^{(7)}, \vect{y}^{(7)}) =$ (you, more)
\item [] $\dots$
\end{itemize}

Notation

\begin{itemize}
\item $\vect{E} \in \mathbb{R}^{d,|\Vocab|}$ is the matrix of word embeddings.
\vspace{0.3cm}
\item $\vect{x}^{(t)} \in \mathbb{R}^{|\Vocab|}$ is one-hot word vector at time step $t$.
\vspace{0.3cm}
\item $\vect{y}^{(t)} \in \mathbb{R}^{|\Vocab|}$ is the ground truth at time step $t$ (also an one-hot word vector).
\end{itemize}

The language model is similar as the RNN described above. It is defined by the following equations:

\begin{equation}
\vect{e}^{(t)} = \vect{E}\vect{x}^{(t)}
\end{equation}
\vspace{0.2cm}
 \begin{equation}
\vect{h}^{(t)} = \sigma(\vect{W}\vect{h}^{(t-1)}+ \vect{U}\vect{e}^{(t)}+ \vect{b})
\end{equation}
\vspace{0.2cm}
\begin{equation}
\vect{\hat{y}}^{(t)} = softmax(\vect{V}\vect{h}^{(t)} + \vect{c})
\end{equation}

\begin{figure}[ht!]
\centering

\scalebox{0.62}{
\begin{tikzpicture}[auto]


% RNN state cell =============================
\node[state] (hprime) {$\vect{h}^{\prime}$};
\node[op, below=15pt of hprime] (a) {$\vect{a}$};

\node[op, right=25pt of a] (b) {$\vect{b}$};
\node[noop, below left=10pt and 10pt of a] (noop-left) {};
\node[state, below left=12pt and 0.1pt of noop-left] (h) {$\vect{h}$};
\node[op, above left=8pt and 15pt of noop-left] (W) {$\vect{W}$};

\node[noop, below right=10pt and 10pt of a] (noop-right) {};
\node[op, below right=12pt and 0.05pt of noop-right] (U) {$\vect{U}$};


% Input layer =============================
\node[op, below=75pt of a] (e) {$\vect{e}$};
\node[op, below left=10pt and 10pt of e] (x) {$\vect{x}$};
\node[op, below right=10pt and 10pt of e] (E) {$\vect{E}$};

%% namedscope
\begin{scope}[on background layer]
\coordinate (p1) at (hprime.north);
\coordinate (p2) at (U.south east);
\coordinate (p3) at (h.south west);
\tkzCircumCenter(p1,p2,p3)
\tkzGetPoint{O}
\tkzDrawCircle[draw=orange, line width=1.5pt, fill=orange!60](O,p1)
\end{scope}

% edges

\path[tedge] (a) edge node[below right= -7pt] {$\;\; \LARGE{\sigma}$}  (hprime);
\path[tedge] (b) -- (a);
\path[tedge] (noop-left) -- (a);
\path[tedge] (noop-right) -- (a);
\path[tedge] (W) -- (noop-left);
\path[tedge] (h.north) to [bend left, out=70, distance=5pt] (noop-left.west);
\path[tedge] (U.north) to [bend right, out=-50, distance=5pt] (noop-right.east);
\path[tedge] (x) -- (e);
\path[tedge] (E) -- (e);
\path[tedge] (e) -- (noop-right);

% RNN output cell =============================

% operations
\node[noop, above=25pt of hprime] (noop-center) {};
\node[op, above=50pt of hprime] (o) {$\vect{o}$};
\node[op, right=15pt of o] (c) {$\vect{c}$};
\node[op, above=15pt of o] (yhat) {$\hat{\textbf{y}}$};
\node[op, below left=1pt and 10pt of o] (V) {$\vect{V}$};

% paths
\path[tedge] (hprime) -- (noop-center);
\path[tedge] (o) edge node[below right= -10pt] {$\;\;$softmax} (yhat);
\path[tedge] (c) -- (o);
\path[tedge] (V) -- (noop-center);
\path[tedge] (noop-center) -- (o);

% namedscope
\begin{scope}[on background layer]
\coordinate (p4) at (yhat.north);
\coordinate (p5) at (V.south west);
\coordinate (p6) at (c.east);
\tkzCircumCenter(p4,p5,p6)
\tkzGetPoint{O}
\tkzDrawCircle[draw=orange, line width=1.5pt, fill=orange!60](O,p4)
\end{scope}

%% namedscope
\begin{scope}[on background layer]
\coordinate (p7) at (e.north);
\coordinate (p8) at (E.east);
\coordinate (p9) at (x.west);
\tkzCircumCenter(p7,p8,p9)
\tkzGetPoint{O}
\tkzDrawCircle[draw=orange, line width=1.5pt, fill=orange!60](O,p7)
\end{scope}

%% Info
\node[textonly, below right=20pt and 1pt of E] (inv1) {};
\node[textonly, above right=20pt and 1pt of E] (inv2) {};
\node[textonly, right=70pt of E] (embedding) {Embedding layer};


% % info edges
\draw[orange!120, line width=1mm]  (embedding) to [out=-180,in=0] (inv1);
\draw[orange!120, line width=1mm] (embedding) to [out=-180,in=0] (inv2);

\end{tikzpicture}
} % scalebox
\end{figure}



At each time $t$ the point-wise loss is:

\vspace{0.2cm}

\begin{align}
L^{(t)} &= CE(\vect{y}^{(t)},\vect{\hat{y}}^{(t)})\\
    &= - \log(\vect{\hat{y}}_{j^{*}})\\
        &= - \log P(x^{(t+1)} = \text{word}_{j^{*}}|x^{(1)}, \dots, x^{(t)})
\end{align}

The loss $L$ is the mean of all the point-wise losses
\begin{equation}
L=\frac{1}{T}\sum_{t=1}^{T}L^{(t)}
\end{equation}


Evaluating a language model. We can evaluate a  language model using a \textit{extrinsic evaluation}: How our model perform in a NLP task such as text auto-completion. Or a \textit{intrinsic evaluation}: Perplexity (PP) can be thought as the weighted average branching factor of a language.


Given $C= x_1, x_2, \dots, x_T$, we define the perplexity of $C$ as:

\begin{align}
PP(C) &= P(x_1, x_2, \dots, x_T)^{-\frac{1}{T}}\\
    & \\
      &= \sqrt[T]{\frac{1}{P(x_1, x_2, \dots, x_T)}}\\
      & \\
      &= \sqrt[T]{\prod_{i=1}^{T}\frac{1}{P(x_i \vert x_1,\dots, x_{i-1})}}
\end{align}

we can relate Loss and Perplexity:

Since
\begin{align}
L^{(t)} & = - \log P(x^{(t+1)} |x^{(1)}, \dots, x^{(t)})\\
& =  \log(\frac{1}{P(x^{(t+1)}|x^{(1)}, \dots, x^{(t)})})\\
\end{align}
We have that:

\begin{align}
        L &=\frac{1}{T} \sum_{t=1}^{T} L^{(t)}\\
          &= \log\left( \sqrt[T]{\prod_{i=1}^{T}\frac{1}{P(x_i \vert x_1,\dots, x_{i-1})}} \right)\\
          &= \log(PP(C))
\end{align}

So another definition of perplexity is

\begin{equation}
2^{L} = PP(C)
\end{equation}





\section{Encoder Decoder}



\section{Atention}


% \begin{figure}[H]
\label{LSTM_cell}
\centering

\scalebox{0.75}{
\begin{tikzpicture}[auto]

% RNN state cell =============================
\node[state] (h) {$\vect{h}$};
\node[op, above right=25pt and 10pt of h] (add1) {$add(\vect{W}_{f},\vect{U}_{f})$};
\node[op,right=25pt of add1] (add2) {$add(\vect{W}_{i},\vect{U}_{i})$};
\node[op,right=25pt of add2] (add3) {$add(\vect{W}_{\tilde{c}},\vect{U}_{\tilde{c}})$};
\node[op,right=25pt of add3] (add4) {$add(\vect{W}_{o},\vect{U}_{o})$};
\node[op,above=20pt of add1] (f) {$\vect{f}$};
\node[op,above=20pt of add2] (i) {$\vect{i}$};
\node[op,above=20pt of add3] (tildec) {$\vect{\tilde{c}}$};
\node[op,above=20pt of add4] (o) {$\vect{o}$};
\node[op,above=20pt of f] (haddamar1) {$\circ$};
\node[state,above left=20pt and 20pt of haddamar1] (C) {$\vect{C}$};
\node[op,above right=20pt and 20pt of i] (haddamar2) {$\circ$};
\node[state,above left=20pt and 20pt of haddamar2] (Cprime) {$\vect{C}^{\prime}$};
\node[op, below right=3pt and 70pt of Cprime] (tanh) {$tanh$};
\node[state, right=50pt of o] (hprime) {$\vect{h}^{\prime}$};
\node[textonly,right=180pt of h] (inv1) {};
\node[op, below=30pt of inv1] (x) {$\vect{x}$};
\node[textonly, below=2pt of Cprime] (name1) {$+$};
\node[textonly, below=2pt of hprime] (name2) {$\circ$};




%% namedscope
\node[draw=orange, fit= (C) (h) (hprime), line width=1.5pt, fill=orange!60, fill opacity=0.2] {};




% edges
\path[tedge] (x) to (add1);
\path[tedge] (x)--(add2);
\path[tedge] (x)--(add3);
\path[tedge] (x)--(add4);
\path[tedge] (h)--(add1);
\path[tedge] (h) to [in=280, distance=5pt] (add2);
\path[tedge] (h)to [in=280, distance=5pt](add3);
\path[tedge] (h)to [in=280, distance=5pt](add4);
\path[tedge] (add1) edge node[below right= -7pt] {$\;\; \Large{\sigma}$}  (f);
\path[tedge] (add2) edge node[below right= -7pt] {$\;\; \Large{\sigma}$}  (i);
\path[tedge] (add3) edge node[below right= -7pt] {$\;\; \Large{tanh}$}  (tildec);
\path[tedge] (add4) edge node[below right= -7pt] {$\;\; \Large{\sigma}$}  (o);
\path[tedge] (f)--(haddamar1);
\path[tedge] (haddamar1)--(Cprime);
\path[tedge] (C)--(haddamar1);
\path[tedge] (i)--(haddamar2);
\path[tedge] (tildec)--(haddamar2);
\path[tedge] (Cprime)--(tanh);
\path[tedge] (haddamar2)--(Cprime);
\path[tedge] (tanh)--(hprime);
\path[tedge] (tanh)--(hprime);
\path[tedge] (o)--(hprime);

\end{tikzpicture}
} % scalebox
\caption{LSTM hidden cell}
\end{figure}


% \begin{figure}[ht!]
\centering

\scalebox{1.40}{
\begin{tikzpicture}[auto]

% RNN state cell =============================
\node[state] (h) {$\vect{h}$};
\node[op, below=30pt of h] (x) {$\vect{x}$};
\node[state, above left=10pt and 10pt of h] (C) {$\vect{C}$};
\node[textonly, above right=5pt and 5pt of h] (inv1) {};
\node[op, above=50pt of h] (yhat) {$\hat{\vect{y}}$};
\node[textonly, right=55pt of C] (inv2) {};
\node[textonly, right=30pt of h] (inv3) {};



%% namedscope LSTM cell
\begin{scope}[on background layer]
\coordinate (p1) at (inv1.north);
\coordinate (p2) at (h.south east);
\coordinate (p3) at (C.north west);
\tkzCircumCenter(p1,p2,p3)
\tkzGetPoint{O}
\tkzDrawCircle[draw=orange, line width=1.5pt, fill=orange!60](O,p1)
\end{scope}


% edges
\path[tedge] (x) -- (h) ;
\path[tedge] (h) -- (yhat);
\path[tedge] (h) -- (inv3);
\path[tedge] (C) -- (inv2);


\end{tikzpicture}
} % scalebox
\end{figure}


% % RNN STATE CELL ====================================
\newcommand{\rnnSimple}[4]{

% operations
\node[state, minimum size=30pt,#4] (h#3) {$\vect{h}^{#1}$};
\node[state, minimum size=30pt,above left=10pt and 10pt of h#3] (C#3) {$\vect{C}^{#1}$};
\node[textonly, above right=5pt and 5pt of h#3] (inv#3) {};
\node[op, minimum size=40pt,below=30pt of h#3] (x#3) {$\vect{x}^{#1}$};
\node[op, minimum size=40pt, above=50pt of h#3] (yhat#3) {$\hat{\vect{y}}^{#1}$};


%% namedscope LSTM cell
\begin{scope}[on background layer]
\coordinate (p1) at (inv#3.north);
\coordinate (p2) at (h#3.south east);
\coordinate (p3) at (C#3.north west);
\tkzCircumCenter(p1,p2,p3)
\tkzGetPoint{O}
\tkzDrawCircle[draw=orange, line width=1.5pt, fill=orange!60](O,p1)
\end{scope}


% edges
\path[tedge] (x#3) -- (h#3);
\path[tedge] (h#3)-- (yhat#3);
}

\begin{figure}[ht!]
\hspace*{-1.0cm}
\scalebox{0.9}{
\begin{tikzpicture}[auto]

% timestep 1
\rnnSimple{(1)}{(0)}{t1}{}

% % timestep 0
\node[state, minimum size=30pt,left=50pt of ht1] (ht0) {$\vect{h}^{(0)}$};
\node[state, minimum size=30pt,left=17pt of Ct1] (Ct0) {$\vect{C}^{(0)}$};

% % timestep 2
\rnnSimple{(2)}{(1)}{t2}{right=50pt of ht1};


% % timestep 2
\rnnSimple{(3)}{(1)}{t3}{right=50pt of ht2};


% % state transfers
\path[tedge] (ht0) -- (ht1);
\path[tedge] (ht1) -- (ht2);
\path[tedge] (ht2) -- (ht3);
\path[tedge] (Ct0) -- (Ct1);
\path[tedge] (Ct1) -- (Ct2);
\path[tedge] (Ct2) -- (Ct3);



\end{tikzpicture}
}%\scalebox
\end{figure}



% \begin{figure}[ht!]
\centering

\scalebox{0.62}{
\begin{tikzpicture}[auto]


% RNN state cell =============================
\node[state] (hprime) {$\vect{h}^{\prime}$};
\node[op, below=15pt of hprime] (a) {$\vect{a}$};

\node[op, right=25pt of a] (b) {$\vect{b}$};
\node[noop, below left=10pt and 10pt of a] (noop-left) {};
\node[state, below left=12pt and 0.1pt of noop-left] (h) {$\vect{h}$};
\node[op, above left=8pt and 15pt of noop-left] (W) {$\vect{W}$};

\node[noop, below right=10pt and 10pt of a] (noop-right) {};
\node[op, below right=12pt and 0.05pt of noop-right] (U) {$\vect{U}$};


% Input layer =============================
\node[op, below=75pt of a] (e) {$\vect{e}$};
\node[op, below left=10pt and 10pt of e] (x) {$\vect{x}$};
\node[op, below right=10pt and 10pt of e] (E) {$\vect{E}$};

%% namedscope
\begin{scope}[on background layer]
\coordinate (p1) at (hprime.north);
\coordinate (p2) at (U.south east);
\coordinate (p3) at (h.south west);
\tkzCircumCenter(p1,p2,p3)
\tkzGetPoint{O}
\tkzDrawCircle[draw=orange, line width=1.5pt, fill=orange!60](O,p1)
\end{scope}

% edges

\path[tedge] (a) edge node[below right= -7pt] {$\;\; \LARGE{\sigma}$}  (hprime);
\path[tedge] (b) -- (a);
\path[tedge] (noop-left) -- (a);
\path[tedge] (noop-right) -- (a);
\path[tedge] (W) -- (noop-left);
\path[tedge] (h.north) to [bend left, out=70, distance=5pt] (noop-left.west);
\path[tedge] (U.north) to [bend right, out=-50, distance=5pt] (noop-right.east);
\path[tedge] (x) -- (e);
\path[tedge] (E) -- (e);
\path[tedge] (e) -- (noop-right);

% RNN output cell =============================

% operations
\node[noop, above=25pt of hprime] (noop-center) {};
\node[op, above=50pt of hprime] (o) {$\vect{o}$};
\node[op, right=15pt of o] (c) {$\vect{c}$};
\node[op, above=15pt of o] (yhat) {$\hat{\textbf{y}}$};
\node[op, below left=1pt and 10pt of o] (V) {$\vect{V}$};

% paths
\path[tedge] (hprime) -- (noop-center);
\path[tedge] (o) edge node[below right= -10pt] {$\;\;$softmax} (yhat);
\path[tedge] (c) -- (o);
\path[tedge] (V) -- (noop-center);
\path[tedge] (noop-center) -- (o);

% namedscope
\begin{scope}[on background layer]
\coordinate (p4) at (yhat.north);
\coordinate (p5) at (V.south west);
\coordinate (p6) at (c.east);
\tkzCircumCenter(p4,p5,p6)
\tkzGetPoint{O}
\tkzDrawCircle[draw=orange, line width=1.5pt, fill=orange!60](O,p4)
\end{scope}

%% namedscope
\begin{scope}[on background layer]
\coordinate (p7) at (e.north);
\coordinate (p8) at (E.east);
\coordinate (p9) at (x.west);
\tkzCircumCenter(p7,p8,p9)
\tkzGetPoint{O}
\tkzDrawCircle[draw=orange, line width=1.5pt, fill=orange!60](O,p7)
\end{scope}

%% Info
\node[textonly, below right=20pt and 1pt of E] (inv1) {};
\node[textonly, above right=20pt and 1pt of E] (inv2) {};
\node[textonly, right=70pt of E] (embedding) {Embedding layer};


% % info edges
\draw[orange!120, line width=1mm]  (embedding) to [out=-180,in=0] (inv1);
\draw[orange!120, line width=1mm] (embedding) to [out=-180,in=0] (inv2);

\end{tikzpicture}
} % scalebox
\end{figure}


% \begin{figure}[ht!]
\centering

\scalebox{1.40}{
\begin{tikzpicture}[auto]

% RNN state cell =============================
\node[state] (h) {$\vect{h}$};
\node[op, below=30pt of h] (e) {$\vect{e}$};
\node[op, above=30pt of h] (yhat) {$\hat{\vect{y}}$};



% edges
\path[tedge] (e) edge node[below right= -4pt] {$\vect{U}$}  (h) ;
\path[tedge] (h) edge [out=-400,in=-320,looseness=8, distance=125pt] node[above right] {$\vect{W}$} (h);
\path[tedge] (h) edge node[below right = -4pt] {$\vect{V}$} (yhat);


\end{tikzpicture}
} % scalebox
\end{figure}


% % RNN STATE CELL ====================================
\newcommand{\rnnSimpleU}[4]{

% operations
\node[state, minimum size=40pt,#4] (h#3) {$\vect{h}^{#1}$};
\node[op, minimum size=40pt, above=30pt of h#3] (yhat#3){$\hat{\vect{y}}^{#1}$};
\node[op, minimum size=40pt,below=30pt of h#3] (e#3) {$\vect{e}^{#1}$};
\node[textonly, below=0.1pt of e#3] {{\Large#2}};

% edges
\path[tedge] (e#3) edge node[below right= -4pt] {$\vect{U}$} (h#3);
\path[tedge] (h#3) edge node[below right = -4pt] {$\vect{V}$} (yhat#3);
}

\begin{figure}[ht!]
\hspace*{-1.0cm}
\scalebox{0.8}{
\begin{tikzpicture}[auto]

% timestep 1
\rnnSimpleU{(1)}{Yes}{t1}{}

% % timestep 0
\node[state, minimum size=40pt,left=50pt of ht1] (ht0) {$\vect{h}^{(0)}$};

% % timestep 2
\rnnSimpleU{(2)}{here}{t2}{right=50pt of ht1};


% % timestep 2
\rnnSimpleU{(3)}{we}{t3}{right=50pt of ht2};


% % state transfers
\path[tedge] (ht0) edge node[above right = 2pt] {$\vect{W}$} (ht1);
\path[tedge] (ht1) edge node[above right = 2pt] {$\vect{W}$} (ht2);
\path[tedge] (ht2) edge node[above right = 2pt] {$\vect{W}$} (ht3);

% % text
\node[textonly, above=40pt of yhatt2] (result) {{\Large $P(x^{(4)}| \text{Yes, here, we})$}};

% Arrow to result
\draw[->, line width=1mm] [bend right, out=-50, distance=25pt](yhatt3.north) to  (result.east);


\end{tikzpicture}
}%\scalebox
\end{figure}



% \begin{figure}[H]
\label{RNNGraphExpanded}
\centering

\scalebox{0.75}{
\begin{tikzpicture}[auto]

% RNN state cell =============================
\node[state] (hprime) {$\vect{h}^{\prime}$};
\node[op, below=15pt of hprime] (a) {$\vect{a}$};

\node[op, right=25pt of a] (b) {$\vect{b}$};
\node[noop, below left=10pt and 10pt of a] (noop-left) {};
\node[state, below left=12pt and 0.1pt of noop-left] (h) {$\vect{h}$};
\node[op, above left=8pt and 15pt of noop-left] (W) {$\vect{W}$};

\node[noop, below right=10pt and 10pt of a] (noop-right) {};
\node[op, below right=12pt and 0.05pt of noop-right] (U) {$\vect{U}$};


% outer operations
\node[op, below=75pt of a] (x) {$\vect{x}$};

%% namedscope
\begin{scope}[on background layer]
\coordinate (p1) at (hprime.north);
\coordinate (p2) at (U.south east);
\coordinate (p3) at (h.south west);
\tkzCircumCenter(p1,p2,p3)
\tkzGetPoint{O}
\tkzDrawCircle[draw=orange, line width=1.5pt, fill=orange!60](O,p1)
\end{scope}

% edges

\path[tedge] (a) edge node[below right= -7pt] {$\;\; \Large{\sigma}$}  (hprime);
\path[tedge] (b) -- (a);
\path[tedge] (noop-left) -- (a);
\path[tedge] (noop-right) -- (a);
\path[tedge] (W) -- (noop-left);
\path[tedge] (h.north) to [bend left, out=70, distance=5pt] (noop-left.west);
\path[tedge] (U.north) to [bend right, out=-50, distance=5pt] (noop-right.east);
\path[tedge] (x) -- (noop-right);

% RNN output cell =============================

% operations
\node[noop, above=25pt of hprime] (noop-center) {};
\node[op, above=50pt of hprime] (o) {$\vect{o}$};
\node[op, right=15pt of o] (c) {$\vect{c}$};
\node[op, above=15pt of o] (yhat) {$\hat{\textbf{y}}$};
\node[op, below left=1pt and 10pt of o] (V) {$\vect{V}$};

% paths
\path[tedge] (hprime) -- (noop-center);
\path[tedge] (o) edge node[below right= -10pt] {$\;\;$softmax} (yhat);
\path[tedge] (c) -- (o);
\path[tedge] (V) -- (noop-center);
\path[tedge] (noop-center) -- (o);

% namedscope
\begin{scope}[on background layer]
\coordinate (p4) at (yhat.north);
\coordinate (p5) at (V.south west);
\coordinate (p6) at (c.east);
\tkzCircumCenter(p4,p5,p6)
\tkzGetPoint{O}
\tkzDrawCircle[draw=orange, line width=1.5pt, fill=orange!60](O,p4)
\end{scope}

\end{tikzpicture}
} % scalebox
\caption{Graph of a RNN}
\end{figure}





% % RNN STATE CELL ====================================
\newcommand{\rnnstatecellU}[4]{

% operations
\node[state, minimum size=40pt,#4] (hprime#3) {$\textbf{h}^{#1}$};
\node[op, minimum size=40pt,below=15pt of hprime#3] (a#3) {$\textbf{a}^{#1}$};

\node[noop, below left=10pt and 10pt of a#3] (noop-left#3) {};
\node[state, minimum size=40pt, below left=12pt and 0.1pt of noop-left#3] (h#3) {$\textbf{h}^{#2}$};
\node[op, above left=8pt and 15pt of noop-left#3] (W#3) {W};

\node[noop, below right=10pt and 10pt of a#3] (noop-right#3) {};
\node[op, below right=12pt and 0.05pt of noop-right#3] (U#3) {U};

% outer operations
\node[op, minimum size=40pt, below=75pt of a#3] (x#3) {$\textbf{x}^{#1}$};

%% namedscope
\begin{scope}[on background layer]
\coordinate (p1#3) at (hprime#3.north);
\coordinate (p2#3) at (U#3.south east);
\coordinate (p3#3) at (h#3.south west);
\tkzCircumCenter(p1#3,p2#3,p3#3)
\tkzGetPoint{O}
\tkzDrawCircle[draw=orange, line width=1.5pt, fill=orange!60](O,p1#3)
\end{scope}

% edges
\path[tedge] (a#3) -- (hprime#3);
\path[tedge] (noop-left#3) -- (a#3);
\path[tedge] (noop-right#3) -- (a#3);
\path[tedge] (W#3) -- (noop-left#3);
\path[tedge] (h#3.north) to [bend left, out=70, distance=5pt] (noop-left#3.west);
\path[tedge] (U#3.north) to [bend right, out=-50, distance=5pt] (noop-right#3.east);
\path[tedge] (x#3) -- (noop-right#3);
}

% RNN OUTPUT ====================================
\newcommand{\rnnoutput}[3]{

% operations
\node[op, minimum size=40pt, #3] (o#2) {$\textbf{o}^{#1}$};
\node[op, minimum size=40pt, above=15pt of o#2] (yhat#2) {$\hat{\textbf{y}}^{#1}$};
\node[op, below left=1pt and 10pt of o#2] (V#2) {$\textbf{V}$};
\node[op, minimum size=40pt, left=50pt of yhat#2] (L#2) {$\textbf{L}^{#1}$};
\node[op, minimum size=40pt, left=20pt of L#2] (y#2) {$\textbf{y}^{#1}$};

% paths
\path[tedge] (o#2) -- (yhat#2);
\path[tedge] (yhat#2) -- (L#2);
\path[tedge] (y#2) -- (L#2);
\path[tedge] (V#2.north) to [bend left, out=90, in=135, distance=10pt] (o#2.west);

% namedscope
\begin{scope}[on background layer]
\coordinate (p1#2) at (yhat#2.north);
\coordinate (p2#2) at (V#2.south west);
\coordinate [right=35pt of o#2] (p3#2);
\tkzCircumCenter(p1#2,p2#2,p3#2)
\tkzGetPoint{O}
\tkzDrawCircle[draw=orange,line width=1.5pt, fill=orange!60](O,p1#2)
\end{scope}
}

\begin{figure}[H]
\hspace*{-1.0cm}
\scalebox{0.425}{
\begin{tikzpicture}[auto]

% timestep 1 cell
\rnnstatecellU{(1)}{(0)}{t1}{}

% timestep 2 cell
\rnnstatecellU{(2)}{(1)}{t2}{right=140pt of hprimet1};

% previous cell
\rnnstatecellU{(\tau-1)}{(\tau-2)}{prev}{right=200pt of hprimet2};

% next cell
\rnnstatecellU{(\tau)}{(\tau-1)}{next}{right=140pt of hprimeprev}

% state transfers
\path[tedge] (hprimet1.east) to [out=10, in=-160] (ht2.west);

\coordinate[right=30pt of hprimet2] (hppp);
\path[tedge] (hppp.east) to [out=10, in=-160] (hprev.west);

\path[tedge] (hprimeprev.east) to [out=10, in=-160] (hnext.west);

% ...
\node[fill=white, right=70pt of at2] (ppp) {\Huge{$\cdots$}};

% output
\rnnoutput{(\tau)}{output}{above=50pt of hprimenext};

\path[tedge] (hprimenext) -- (ooutput);

\end{tikzpicture}
}%\scalebox
\caption{The computational graph to compute the training loss of a RNN}
\end{figure}




% \newcommand{\rnnstatecell}[4]{

% operations
\node[state, minimum size=40pt,#4] (hprime#3) {$\vect{h}^{#1}$};
\node[op, minimum size=40pt,below=15pt of hprime#3] (a#3) {$\vect{a}^{#1}$};

\node[noop, below left=10pt and 10pt of a#3] (noop-left#3) {};
\node[state, minimum size=40pt, below left=12pt and 0.1pt of noop-left#3] (h#3) {$\vect{h}^{#2}$};
\node[op, above left=8pt and 15pt of noop-left#3] (W#3) {$\vect{W}$};

\node[noop, below right=10pt and 10pt of a#3] (noop-right#3) {};
\node[op, below right=12pt and 0.05pt of noop-right#3] (U#3) {$\vect{U}$};

% outer operations
\node[op, minimum size=40pt, below=75pt of a#3] (x#3) {$\vect{x}^{#1}$};

%% namedscope
\begin{scope}[on background layer]
\coordinate (p1) at (hprime#3.north);
\coordinate (p2) at (U#3.south east);
\coordinate (p3) at (h#3.south west);
\tkzCircumCenter(p1,p2,p3)
\tkzGetPoint{O}
\tkzDrawCircle[draw=orange,line width=2pt, fill=orange!60](O,p1)
\end{scope}

% edges
\path[tedge] (a#3) -- (hprime#3);
\path[tedge] (noop-left#3) -- (a#3);
\path[tedge] (noop-right#3) -- (a#3);
\path[tedge] (W#3) -- (noop-left#3);
\path[tedge] (h#3.north) to [bend left, out=70, distance=5pt] (noop-left#3.west);
\path[tedge] (U#3.north) to [bend right, out=-50, distance=5pt] (noop-right#3.east);
\path[tedge] (x#3) -- (noop-right#3);
}

\begin{figure}
\centering

\scalebox{0.42}{
\begin{tikzpicture}[auto]

% previous cell
\rnnstatecell{(\tau-1)}{(\tau-2)}{prev}{}

% past cells
\node[textonly, left=60pt of Wprev] (dots) {{\Huge $\cdots$}};

% past edge
\path[tedge] (dots) to [out=10, in=-160] (hprev.west);


% next cell
\rnnstatecell{(\tau)}{(\tau-1)}{next}{right=230pt of hprimeprev}

% state transfer
\path[tedge] (hprimeprev.east) to [out=10, in=-160] (hnext.west);

% loss
\node[op, minimum size=40pt, above right=80pt and 40pt of hprimenext] (L) {$L^{(\tau)}$};
\node[textonly, above=80pt of hprimenext] (ppp) {{\Huge $\cdots$}};

% loss edges
\path[tedge] (hprimenext.north) to [bend left, out=-110, in=100, distance=20pt] (ppp.west);
\path[tedge] (ppp) -- (L);

% gradients from time step tau
\node[gradient, above right=1pt and 70pt of hprimenext] (grad-hprime) {$\grad{\vect{h}^{(\tau)}}{L^{(\tau)}}$};
\node[textonly, right=0.1pt of grad-hprime] {$=\dotpPright{\vect{V}}{\hat{\vect{y}}^{(\tau)} - \vect{y}^{(\tau)}}$};
\node[gradient, right=65pt of anext] (grad-a) {$\grad{\vect{a}^{(\tau)}}{L^{(\tau)}}$};
\node[textonly, right=0.1pt of grad-a] {$=\left(\grad{\vect{h}^{(\tau)}}{L^{(\tau)}}\right)\circ\vect{h}^{(\tau)}\circ(1-\vect{h}^{(\tau)})$};
\node[gradient, below right=0.1pt and 40pt of Unext] (grad-U) {$\grad{\vect{U}^{(\tau)}}{L^{(\tau)}}$};
\node[textonly, right=0.1pt of grad-U] {$=\outerp{\grad{\vect{a}^{(\tau)}}{L^{(\tau)}}}{\vect{x}^{(\tau)}}$};
\node[gradient, above left=70pt and 10pt of Wnext] (grad-W) {$\grad{\vect{W}^{(\tau)}}{L^{(\tau)}}$};
\node[textonly, left=0.1pt of grad-W] {$\outerp{\grad{\vect{a^{(\tau)}}}{L^{(\tau)}}}{\vect{h}^{(\tau-1)}}=$};


% gradients from time step tau -1
\node[gradient, above left=60pt and 10pt of hprimeprev] (grad-hprimeprev) {$\grad{\vect{h}^{(\tau-1)}}{L^{(\tau)}}$};
\node[textonly, left=0.1pt of grad-hprimeprev] {$\dotpPright{\vect{W}}{\grad{\vect{a}^{(\tau)}}{L^{(\tau)}}}=$};


\node[gradient, below left=55pt and -20pt of hnext] (grad-a-prev) {$\grad{\vect{a}^{(\tau-1)}}{L^{(\tau)}}$};
\node[textonly, right=0.1pt of grad-a-prev] {$=\left(\grad{\vect{h}^{(\tau-1)}}{L^{(\tau)}}\right)\circ\vect{h}^{(\tau-1)}\circ(1-\vect{h}^{(\tau-1)})$};


\node[gradient, below left=20pt and 10pt of grad-a-prev] (grad-U-prev) {$\grad{\vect{U}^{(\tau-1)}}{L^{(\tau)}}$};
\node[textonly, left=0.1pt of grad-U-prev] {$\outerp{\grad{\vect{a}^{(\tau-1)}}{L^{(\tau)}}}{\vect{x}^{(\tau-1)}}=$};


\node[gradient, above left=70pt and 10pt of Wprev] (grad-W-prev) {$\grad{\vect{W}^{(\tau-1)}}{L^{(\tau)}}$};
\node[textonly, left=0.1pt of grad-W-prev] {$\outerp{\grad{\vect{a^{(\tau-1)}}}{L^{(\tau)}}}{\vect{h}^{(\tau-2)}}=$};

\node[gradient, below left=20pt and 10pt of hprev] (grad-hprev) {$\grad{\vect{h}^{(\tau-2)}}{L^{(\tau)}}$};
\node[textonly, left=0.1pt of grad-hprev] {$\dotpPright{\vect{W}}{\grad{\vect{a}^{(\tau-1)}}{L^{(\tau)}}}=$};



% gradient's edges
\path[tedge] (hprimenext) -- (grad-hprime);
\path[tedge] (anext) -- (grad-a);
\path[tedge] (Unext) -- (grad-U);
\path[tedge] (Wnext) -- (grad-W);
\path[tedge] (hprimeprev) -- (grad-hprimeprev);
\path[tedge] (aprev.east) to [out=10, in=-160] (grad-a-prev);
\path[tedge] (Uprev) -- (grad-U-prev);
\path[tedge] (Wprev) -- (grad-W-prev);
\path[tedge] (hprev) -- (grad-hprev);


\end{tikzpicture}
}%\scalebox
\end{figure}



% \begin{figure}[ht!]
\centering

\scalebox{0.71}{
\begin{tikzpicture}[auto]


% RNN state cell =============================
\node[state] (hprime) {$\myprime{\vect{h}}$};
\node[op, below=15pt of hprime] (a) {$\vect{a}$};

\node[noop, below left=10pt and 10pt of a] (noop-left) {};
\node[state, below left=12pt and 0.1pt of noop-left] (h) {$\vect{h}$};
\node[op, above left=8pt and 15pt of noop-left] (W) {$\vect{W}$};

\node[noop, below right=10pt and 10pt of a] (noop-right) {};
\node[op, below right=12pt and 0.05pt of noop-right] (U) {$\vect{U}$};

% gradients
% $\grad{\myprime{\vect{h}}}{L}$
% $\outerp{\grad{\vect{a}}{L}}{\vect{x}}=$

\node[gradient, right=50pt of hprime] (grad-hprime) {$\grad{\myprime{\vect{h}}}{L}$};
\node[textonly, right=0.1pt of grad-hprime] {$=\dotpPright{\vect{V}}{\grad{\vect{o}}{L}}$};

\node[gradient, right=65pt of a] (grad-a) {$\grad{\vect{a}}{L}$};
\node[textonly, right=0.1pt of grad-a] {$=\left(\grad{\myprime{\vect{h}}}{L}\right)\circ\myprime{\vect{h}}\circ(1-\myprime{\vect{h}})$};

\node[gradient, right=40pt of U] (grad-U) {$\grad{\vect{U}}{L}$};
\node[textonly, right=0.1pt of grad-U] {$=\outerp{\grad{\vect{a}}{L}}{\vect{x}}$};

\node[gradient, left=30pt of W] (grad-W) {$\grad{\vect{W}}{L}$};
\node[textonly, left=0.1pt of grad-W] {$\outerp{\grad{\vect{a}}{L}}{\vect{h}}=$};

\node[gradient, left=30pt of h] (grad-h) {$\grad{\vect{h}}{L}$};
\node[textonly, left=0.1pt of grad-h] {$\dotpPright{\vect{W}}{\grad{\vect{a}}{L}}=$};

% outer operations
\node[op, below=75pt of a] (x) {$\vect{x}$};

%% namedscope
\begin{scope}[on background layer]
\coordinate (p1) at (hprime.north);
\coordinate (p2) at (U.south east);
\coordinate (p3) at (h.south west);
\tkzCircumCenter(p1,p2,p3)
\tkzGetPoint{O}
\tkzDrawCircle[draw=orange, line width=1.5pt, fill=orange!60](O,p1)
\end{scope}

% edges
\path[tedge] (hprime) -- (grad-hprime);
\path[tedge] (a) -- (grad-a);
\path[tedge] (U) -- (grad-U);
\path[tedge] (W) -- (grad-W);
\path[tedge] (h) -- (grad-h);

\path[tedge] (a) -- (hprime);
\path[tedge] (noop-left) -- (a);
\path[tedge] (noop-right) -- (a);
\path[tedge] (W) -- (noop-left);
\path[tedge] (h.north) to [bend left, out=70, distance=5pt] (noop-left.west);
\path[tedge] (U.north) to [bend right, out=-50, distance=5pt] (noop-right.east);
\path[tedge] (x) -- (noop-right);

% RNN output cell =============================

% operations
\node[op, above=50pt of hprime] (o) {$\vect{o}$};
\node[op, above=15pt of o] (yhat) {$\hat{\vect{y}}$};
\node[op, below left=1pt and 10pt of o] (V) {$\vect{V}$};
\node[op, right=50pt of yhat] (L) {$L$};
\node[op, right=20pt of L] (y) {$\vect{y}$};

% gradients
\node[gradient, left=20pt of V] (grad-V) {$\grad{\vect{V}}{L}$};
\node[gradient, right=50pt of o] (grad-o) {$\grad{\vect{o}}{L}$};
\node[textonly, right=0.1pt of grad-o] {$=\hat{\vect{y}}-\vect{y}$};
\node[textonly, left=0.1pt of grad-V] {$\outerp{\grad{\vect{o}}{L}}{\myprime{\vect{h}}}=$};

% paths
\path[tedge] (hprime) -- (o);
\path[tedge] (o) -- (yhat);
\path[tedge] (yhat) -- (L);
\path[tedge] (y) -- (L);
\path[tedge] (o) -- (grad-o);
\path[tedge] (V.north) to [bend left, out=90, in=135, distance=10pt] (o.west);
\path[tedge] (V) -- (grad-V);

% namedscope
\begin{scope}[on background layer]
\coordinate (p4) at (yhat.north);
\coordinate (p5) at (V.south west);
\coordinate [right=35pt of o] (p6);
\tkzCircumCenter(p4,p5,p6)
\tkzGetPoint{O}
\tkzDrawCircle[draw=orange, line width=1.5pt, fill=orange!60](O,p4)
\end{scope}

\end{tikzpicture}
} % scalebox
\end{figure}
