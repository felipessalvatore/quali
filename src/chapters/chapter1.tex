\chapter{Dialog Systems}\label{dialog}
\section{intro}\label{sec:intro}

In this PhD thesis we will investigate the problem of \textbf{building a non-task-oriented dialog system}. This problem is central to the field of \textit{artificial intelligence} (AI) due to the seminal paper by Alan Turing \cite{Turing} in which he proposed an imitation game. The idea behind the game was simple: three players A, B and C can interact only by nonpersonal communication. C knows that he will interact with a computer (A) and a person (B), but he does not know which is which. C should exchange some conversation both with A and B; after some time he should guess who is the computer. The goal of A is to be as human as possible in order to fool C; and the goal of B is to help C come to the right answer.

\par This is the famous \textbf{Turing Test}. At that time Turing proposed that test as a way to make more concrete the philosophical question "can machines think?". This can lead to a whole discussion of the nature of thinking and intelligence. As a mature field, AI has distantiate itself from these abstract an general questions and have concentrated in \textit{building agents to solve tasks reserved exclusively to humans}: playing chess, driving a car, cleaning a room, writing a letter to a friend, etc.    

\par The sub-area of AI that deals with problems involving human language is called \textit{natural language processing} (NLP). It concentrate in tasks such text understanding (por mais coisa). The 

\section{Theoretical framework}

to do

\section{Data}

to do

\section{Evaluation}

to do