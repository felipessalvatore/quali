\chapter{Dialog Systems}\label{dialog}
\section{intro}\label{sec:intro}

In this PhD thesis we will investigate the problem of \textbf{building a non-task-oriented dialog system}. This problem is central to the field of \textit{artificial intelligence} (AI) due to the seminal paper by Alan Turing \cite{Turing} in which he proposed an imitation game. The idea behind the game was simple: three players A, B and C can interact only by nonpersonal communication. C knows that he will interact with a computer (A) and a person (B), but he does not know which is which. C should exchange some conversation both with A and B; after some time he should guess who is the computer and who is the human. The goal of A is to be as human as possible in order to fool C; and the goal of B is to help C come to the right answer.

\par This is the famous \textit{Turing Test}. At that time Turing proposed that test as a way to make more concrete the philosophical question "can machines think?". This can lead to a whole discussion of the nature of thinking and intelligence. As a mature field, AI has distantiate itself from these abstract an general questions and have concentrated in \textit{building agents to solve tasks reserved exclusively to humans}: playing chess, driving a car, cleaning a room, writing a letter to a friend, etc.    

\par The sub-area of AI that deals with problems involving human language is called \textit{natural language processing} (NLP). It concentrate in tasks such text understanding (por mais coisa). NLP , as the field of AI as a whole, have change dramatically in the past 10 years. For some time the main paradigm in the field was the so called \textbf{knowledge base approach}: the main goal of AI was to develop intelligent systems capable of solving certain classes of problems by having a \textit{representation} or \textit{model} of the world. In this view, a decision by a machine is synonymous to \textit{inferring in a formal language}\cite{McCarthy}.


\par A different point of view is that AI should construct systems that acquire their knowledge via data observation. 
More useful than programming hand-coded rules in an AI agent, we should enable that agent to extract patters from the complexity of data. This is the \textbf{machine learning approach}. One extension of this view is called \textbf{deep learning} where we learn i) how to map a representation of the data to a desirable output and ii) how to represent the data.

\par After this brief presentation we can state our objectives more clearly: here we will investigate how to build a non-task-oriented dialog system using a deep learning family of models. This family of models is know as \textbf{Recurrent Neural Networks} (RNNs). RNNs have achieve increasing success in a variety of NLP tasks such as machine translation, speech recognition, sentiment analysis, language modeling, etc. Together with this success the community has proposed also a variety of new architectures. We will explore some of these architectures for the task of dialog generation.




\section{Theoretical framework}

\subsection{Machine learning}


\subsection{Neural network}


\subsection{Computational graphs}


\section{Dialog as a learning task}

to do 

\section{Dialog data}

Twiter Ubuntu openSUb

\section{Evaluation}

BLUE, METEOR, others 