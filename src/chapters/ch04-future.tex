\chapter{Future Steps}
\label{ch04:FutureSteps}

The previous chapters resume what we have done so far: we grasp the theoretical framework to formulate this NLP problem; we reviewed the literature on dialog generation; we built a software workbench to perform different experiments; and we have isolated a specific problem not sufficiently not addressed in the literature: logical reasoning for dialog agents.

In July we will present this work at one summer school organized by the company DeepMind in Europe \url{https://tmlss.ro/} and so we expect to gather more feedback.

To address our research proposal we decided to formulate the following next steps:

\begin{itemize}
\item Apply regularization strategies on the available models to overcome the reported overfitting problem. 
\item Finish the Entailment-QA corpus to have a fine grain analysis of the result that we are seeing on the SICK corpus.
\item Explore the different extensions for all mentioned models.
\item Explore new models not mentioned here, like Dynamic Memory Networks \cite{KumarISBEPOGS15} and the models using the Memory Attention and Composition (MAC) cell \cite{Manning18}.
\item There is a different literature that frames the dialog problem as an MDP (Markovian Decision Process) and a POMDP (Partially Observable Markovian Decision Process) applying different techniques of reinforcement learning (a recent example is \cite{Li:2016}). It is fruitful to investigate if these techniques can help our research.
\item One of the main focused here is model comparison. It would be fruitful if we could use the available literature  on the theory of comparing models (e.g.,  \cite{BenavoliCDZ17}) to refine our analysis.
\end{itemize}


\section{Work Plan}
\label{sec:work-plan}

Here, we use a visual tool to display the scheduling of future and past activities. This serve as a sanity check to verify if the
proposed goals can be realistically achieved.

\begin{table}[ht!]
  \center
  \begin{tabular}{|c|c|c|c|c|c|c|c|c|c|}\hline
    & \multicolumn{2}{c|}{2016} & \multicolumn{2}{c|}{2017} & \multicolumn{2}{c|}{2018} & \multicolumn{2}{c|}{2019} & \multicolumn{1}{c|}{2020} \\ \cline{2-10}
    \raisebox{1.5ex}{Activity} & 1st & 2nd & 1st & 2nd & 1st & 2nd & 1st & 2nd & 1st \\ \hline \hline
    Courses & \cellcolor{green!25} & \cellcolor{green!25} &  &  &  &  & & & \\ \hline
    Teaching Assist. (PAE) & & & \cellcolor{green!25} &  &  &  & & & \\ \hline
    Bibliographic Review & \cellcolor{green!25} & \cellcolor{green!25} & \cellcolor{green!25} & \cellcolor{green!25} & \cellcolor{green!25} & \cellcolor{blue!25} & \cellcolor{blue!25} & \cellcolor{blue!25} & \\ \hline 
    Plankton Classification & & & \cellcolor{green!25} & \cellcolor{green!25} &  & & &  & \\ \hline
    Densemap - Gathering & & & &  & \cellcolor{green!25}  & & &  & \\ \hline
    Qualification Writing & & & & & \cellcolor{green!25} & &  & & \\ \hline
    Qualification Exam & & & & & \cellcolor{green!25} & & & & \\ \hline
    Split-Site Double Degree PhD & & & & & & \cellcolor{blue!25} & \cellcolor{blue!25} & & \\ \hline
    Densemap - Scattering & & & &  & & \cellcolor{blue!25} & &  & \\ \hline
    Active Learning Tool & & & &  & &  & \cellcolor{blue!25} &  & \\ \hline
    Data Augmentation Methods & & & &  & &  & & \cellcolor{blue!25} & \\ \hline
    %Paper Writing & & & & \cellcolor{green!25} & \cellcolor{green!25} & \cellcolor{blue!25} & & \cellcolor{blue!25} & \\ \hline
    Thesis Writing & & & & & & & & \cellcolor{blue!25} & \\ \hline
    Thesis Defense & & & & & & & & & \cellcolor{blue!25}  \\ \hline
  \end{tabular}
  \caption{Schedule for the PhD program. Completed activities are shown in green, while future activities are in blue.}
  \label{tab:schedule}
\end{table}