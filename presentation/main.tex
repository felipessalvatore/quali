\documentclass[10pt]{beamer}
\usetheme{metropolis}
% all imports
\input{all_imports}

\AtBeginEnvironment{quote}{\singlespacing}

% new commands
\input{all_new_commands}

% definitions
\input{definitions/colors}
\input{definitions/styles}

\title{Adding semantic robustness to dialog agents} 

\date{\today}

\vspace{1.0 cm}

\author{
  Felipe Salvatore\\
  \url{https://felipessalvatore.github.io/}\vspace{1.8 cm}
}

\institute{\textbf{IME-USP}: Institute of Mathematics and Statistics, University of São Paulo}




\begin{document}


\maketitle


\begin{frame}{Sistemas de diálogo}
 criar um programa capaz de dialogar com ser humano
\begin{center}
\includegraphics[scale=0.16]{images/turing.jpg}
\end{center}
\end{frame}


\begin{frame}{Sistemas de diálogo}
goal -driven vs non-goal driven
\end{frame}

\section{Sistemas de diálogo baseados em redes neurais}

\begin{frame}{Modelos de linguagem baseados em redes neurais}
Nos chamamos de \alert{modelo de linguagem} uma distribuição de probabildiade sobre uma sequencia de tokens em uma lingua natural.

\[
P(x_1,x_2,x_3,x_4) = p
\]

Em vez de usar uma abordagem que seja específica para o domínio da linguagem natural, podemos usar um modelo para predição de dados sequencias:  \textbf{uma rede recorrente (RNN)}. \\

Nossa tarefa de aprendizado é estimar a distribuição de probabilidade

\[
P(x_{n} = \text{palavra}_{j^{*}} | x_{1}, \dots ,x_{n-1})
\]

para qualquer $(n-1)$ sequencia de palavras $x_{1}, \dots ,x_{n-1}$.

\end{frame}

\begin{frame}{O modelo de linguagem com RNN}
\input{tikzfiles/LanguageModelUnfolded}
\end{frame}



\begin{frame}{GRU: Gated Recurrent Units}
\begin{center}
\includegraphics[scale=0.25]{images/gru.png}
\end{center}
\end{frame}


% \begin{frame}{LSTM: Long Short Term Memory}
% \begin{center}
% \includegraphics[scale=0.23]{images/lstm.png}
% \end{center}
% \end{frame}


\begin{frame}{Exemplo: TrumpBot\\\url{https://github.com/felipessalvatore/MyTwitterBot}}
\begin{center}
\includegraphics[scale=0.24]{images/TrumpBot.png}
\end{center}
\end{frame}


\begin{frame}{Exemplo: Funk Generator\\ \url{https://github.com/lucasmoura/funk_generator}}
\begin{quote}
\centering
É o dj que tá tocando e não sabe de nada\\ 
Eu já tô no clima e já tô no meu nome \\
Cordão de ouro no pescoço eu tô na moda \\
Com a camisa da \\
Louis \\
Vuitton \\
Pulo da morena que elas gosta\\ 
E se eu te pego no baile \\
De captiva de citroen ou de hayabusa\\ 
Tu viu a 1100 cilindradas \\
Se eu tô no litoral de cordão de ouro\\ 
De cordão de ouro no pescoço\\
\end{quote}
\end{frame}

\begin{frame}{Seq2seq: diálogo \cite{DBLP:journals/corr/VinyalsL15}}
\input{tikzfiles/seq2seq_dialog}
\end{frame}


\begin{frame}{Seq2seq: tradução \cite{luzfinger2017}}
\input{tikzfiles/Translation}
\end{frame}

\begin{frame}{Exemplo de diálogo \cite{DBLP:journals/corr/VinyalsL15}}
\begin{center}
\includegraphics[scale=0.3]{images/exemplo1.png}
\end{center}
\end{frame}


\section{Métricas}


\begin{frame}{Avaliação humana \cite{Lowe:2016}}
\begin{center}
\includegraphics[scale=0.4]{images/exemploEval1.png}
\end{center}
\end{frame}




\begin{frame}{Avaliação automática: BLEU (bilingual evaluation understudy) \cite{Papineni2001}}
Essa métrica compara n-gramas (até 4) da resposta candidata com os n-gramas da refência da tradução e conta o numero de acertos. Essa métrica também penaliza traudções muito curtas:

\begin{equation}
BLUE(r, \hat{r}) = min \left(1, \frac{len(\hat{r})}{len(r)} \right) \left(\prod_{n=1}^{4} precision_{n}(r, \hat{r}) \right)^{\frac{1}{4}}
\end{equation}
em que $ precision_{n}(r, \hat{r})$ é o número de overlap de $n$ gramas de $r$ e $\hat{r}$ dividido pelo número de todos os $n$-gramas de $\hat{r}$. 

$BLUE(r, \hat{r}) \in [0,1]$

\end{frame}


\begin{frame}{Avaliação automática: problemas}
\begin{center}
\includegraphics[scale=0.23]{images/weak_corr.png}
\end{center}

"In particular, we show that these metrics (BLEU, METEOR, ROUGE) have only a small positive correlation on the chitchat oriented Twitter dataset, and no correlation at all on the technical Ubuntu Dialogue Corpus." \cite{LiuLSNCP16}

\end{frame}

\section{De diálogos abertos para pequenas tarefas}

\begin{frame}{bAbI \cite{WestonBCM15}}
Criar uma série de pequenas tarefas para testar diferentes capacidades de um sistema de diálogo.


\begin{center}
\includegraphics[scale=0.25]{images/babi.png}
\end{center}
\end{frame}

\begin{frame}{modelo de atenção}

\end{frame}


\begin{frame}{modelo de memória}

\end{frame}

\begin{frame}{ParlAI \\ \url{https://github.com/facebookresearch/ParlAI}}

\begin{center}
\includegraphics[scale=0.84]{images/parlai.png}
\end{center}

"ParlAI (pronounced 'par-lay') is a framework for dialog AI research, implemented in Python.

Its goal is to provide researchers:

\begin{itemize}
\item a unified framework for sharing, training and testing dialog models
\item many popular datasets available all in one place, with the ability to multi-task over them
\item seamless integration of Amazon Mechanical Turk for data collection and human evaluation"
\end{itemize}

\end{frame}

\begin{frame}{Experimentos}
\begin{center}
\includegraphics[scale=0.34]{images/comparative_results_babi1.png}
\end{center}
\end{frame}

\begin{frame}{Experimentos}
\begin{center}
\includegraphics[scale=0.34]{images/comparative_results_babi2.png}
\end{center}
\end{frame}




\section{Entailment-QA}

\begin{frame}{bAbI: task 15}

\alert{Basic Deduction}

\begin{center}
\includegraphics[scale=0.28]{images/babi15.png}
\end{center}
\begin{quote} 
\centering 
$P^{1}$ are afraid of $Q^{1}$\\
$P^{2}$ are afraid of $Q^{2}$\\
$P^{3}$ are afraid of $Q^{3}$\\
$P^{4}$ are afraid of $Q^{4}$\\
$c^{1}$ is a $P^{1}$\\
$c^{2}$ is a $P^{2}$\\
$c^{3}$ is a $P^{3}$\\
$c^{4}$ is a $P^{4}$\\
What is $c^j$ afraid of?\\
\end{quote}


\end{frame}


\begin{frame}{bAbI: task 16}

\alert{Basic Induction}

\begin{center}
\includegraphics[scale=0.28]{images/babi16.png}
\end{center}
\begin{quote} 
\centering 
$c^{1}$ is a $P^{1}$\\
$c^{1}$ is $C^{1}$\\
$c^{2}$ is a $P^{2}$\\
$c^{2}$ is $C^{2}$\\
$c^{3}$ is a $P^{3}$\\
$c^{3}$ is $C^{3}$\\
$c^{4}$ is a $P^{4}$\\
$c^{4}$ is $C^{4}$\\
$c$ is a $P^{j}$\\
What color is $c$?\\
\end{quote}

\end{frame}


\begin{frame}{SICK}

\end{frame}

\begin{frame}{quora}

\end{frame}


\begin{frame}{DialogGym}

\begin{center}
\includegraphics[scale=0.58]{images/DGred2.png}
\end{center}

\url{https://github.com/felipessalvatore/DialogGym}

\end{frame}


\begin{frame}{Um novo conjunto de tarefas}
\begin{itemize}
\item \textbf{Task 1: entailment prediction} Given two sentences $p$ and $q$ the agent is asked to detect a basic entailment relation between them, i.e., the agent should respond if $p$ implies $q$, if $p$ contradicts $q$ or if $p$ is neutral to $q$. For example, the sentences "\textit{A man is thinking}" and "\textit{There is no man thinking}" is given to the agent, he needs to detect the quantifier to spot the contradiction between these two informations.
\item \textbf{Task 2: similarity prediction} The agent is questioned to indicate how related are the meaning of two sentences, e.g., "\textit{A man is reading the email. Someone is reading the email. Are the sentences above related?}". There are only 4 possible answers: "not related", "somewhat related", "related", "strongly related". 
\item \textbf{Task 3: paraphrase prediction} The agent is asked (a yes/no question) to identify if two given questions express the same meaning using different words, e.g., "\textit{Who was Pele? Who is Pele? Are the above questions duplicate?}".
\end{itemize}
\end{frame}

\begin{frame}{Primeiros resultados}
\begin{center}
\includegraphics[scale=0.28]{images/both_semantic_tasks.png}
\end{center}
\end{frame}

\begin{frame}{SICK como melhorar}

\end{frame}


\begin{frame}{Entailment-QA}
comentar o sick

\end{frame}


\begin{frame}{Entailment-QA}

\begin{enumerate}
\item \textbf{Boolean Connectives}
\item \textbf{First-Order Quantifiers}
\item \textbf{Synonymy}
\item \textbf{Antinomy}
\item \textbf{Hypernymy}
\item \textbf{Active/Passive voice}
\end{enumerate}
\end{frame}

\begin{frame}{Entailment-QA: task 1}
\begin{itemize}
\item \alert{Entailment} ($s_1$ implies $s_2$)
\begin{itemize}
\item $\underbrace{P^{1}a^1 \land \dots \land P^{n}a^n}_{s_1}, \underbrace{P^{j}a^j}_{s_2}$ 
\item $\underbrace{P^{j}a^j}_{s_1}, \underbrace{P^{1}a^1 \lor \dots \lor P^{n}a^n}_{s_2}$
\item $\underbrace{Pa}_{s_1}, \underbrace{\lnot \lnot Pa}_{s_2}$
\end{itemize}

\vspace{0.4cm}
\item \alert{Not entailment} ($s_1$ does not implies $s_2$)
\begin{itemize}
\item $\underbrace{P^{j}a^j}_{s_1}, \underbrace{P^{1}a^1 \land \dots \land P^{n}a^n}_{s_2}$ 
\item $\underbrace{P^{1}a^1 \lor \dots \lor P^{n}a^n}_{s_1}, \underbrace{P^{j}a^j}_{s_2}$
\item $\underbrace{Pa}_{s_1}, \underbrace{\lnot Pa}_{s_2}$
\end{itemize}
\end{itemize}
\end{frame}


\begin{frame}{Entailment-QA: task 1}
\begin{itemize} 
\item[] Ashley is fit
\item[] Ashley is not fit
\item[] The first sentence implies the second sentence? \alert{A: no}
\end{itemize}

\vspace{0.3cm}


\begin{itemize} 
\item[]Avery is nice and Avery is obedient
\item[]Avery is nice
\item[]The first sentence implies the second sentence? \alert{A: yes}
\end{itemize}

\vspace{0.3cm}

\begin{itemize} 
\item[]Elbert is handsome or Elbert is long
\item[]Elbert is handsome
\item[]The first sentence implies the second sentence? \alert{A: no}
\end{itemize}
\end{frame}


\begin{frame}{Entailment-QA: task 2}

\begin{itemize}
\item \alert{Entailment}
\begin{itemize}
\item $\forall x Px, Pa$ 
\item $Pa, \exists x Px$ 
\end{itemize}
\item \alert{Contradiction}
\begin{itemize}
\item $\forall x Px, \lnot Pa$ 
\item $\forall x Px, \exists x \lnot Px$ 
\end{itemize}
\item \alert{Neutral}
\begin{itemize}
\item $Pa,Qa$ 
\item $\forall x Px, \lnot Qa$ 
\end{itemize}
\end{itemize}
\end{frame}


\begin{frame}{Entailment-QA: task 2}

\begin{itemize} 
\item[] Every person is lively
\item[] Belden is lively
\item[] What is the semantic relation? \alert{A: entailment}
\end{itemize}

\begin{itemize} 
\item[] Every person is short
\item[] There is one person that is not short
\item[] What is the semantic relation?  \alert{A: contradiction}
\end{itemize}

\begin{itemize} 
\item[] Every person is beautiful
\item[] Abilene is not blue
\item[] What is the semantic relation? \alert{A: neutral}
\end{itemize}
\end{frame}



\begin{frame}{Resultados até agora}
\begin{center}
\includegraphics[scale=0.42]{images/comparative_results.png}
\end{center}
\end{frame}



\begin{frame}{Resultados até agora}
\begin{center}
\includegraphics[scale=0.28]{images/training_acc_EntailQA2_mem.png}
\end{center}
\end{frame}


\begin{frame}{Resultados até agora}
\begin{center}
\includegraphics[scale=0.42]{images/cm_mem_EntailQA2.png}
\end{center}
\end{frame}


\begin{frame}{Próximos passos}
\begin{itemize}

\item Terminar as tarefas

\item Melhor o treinamento com os modelos atuais

\item Explorar novos modelos
\end{itemize}


\end{frame}

\begin{frame}[allowframebreaks]{Referências}

  \bibliography{my_references}
  \bibliographystyle{abbrv}

\end{frame}

\end{document}




\end{document}