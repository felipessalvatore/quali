\documentclass[10pt]{beamer}
\usetheme{metropolis}
% all imports
\input{all_imports}

\AtBeginEnvironment{quote}{\singlespacing}

% new commands
\input{all_new_commands}

% definitions
\input{definitions/colors}
\input{definitions/styles}

\title{Adding semantic robustness to dialog agents} 

\date{\today}

\vspace{1.0 cm}

\author{
  Felipe Salvatore\\
  \url{https://felipessalvatore.github.io/}\vspace{1.8 cm}
}

\institute{\textbf{IME-USP}: Institute of Mathematics and Statistics, University of São Paulo}




\begin{document}


\maketitle


\begin{frame}{Sistemas de diálogo}
 criar um programa capaz de dialogar com ser humano
\begin{center}
\includegraphics[scale=0.16]{images/turing.jpg}
\end{center}
\end{frame}


\begin{frame}{Sistemas de diálogo}
goal -driven vs non-goal driven
\end{frame}

\section{Sistemas de diálogo baseados em redes neurais}

\begin{frame}{Modelos de linguagem baseados em redes neurais}
Nos chamamos de \alert{modelo de linguagem} uma distribuição de probabildiade sobre uma sequencia de tokens em uma lingua natural.

\[
P(x_1,x_2,x_3,x_4) = p
\]

Em vez de usar uma abordagem que seja específica para o domínio da linguagem natural, podemos usar um modelo para predição de dados sequencias:  \textbf{uma rede recorrente (RNN)}. \\

Nossa tarefa de aprendizado é estimar a distribuição de probabilidade

\[
P(x_{n} = \text{palavra}_{j^{*}} | x_{1}, \dots ,x_{n-1})
\]

para qualquer $(n-1)$ sequencia de palavras $x_{1}, \dots ,x_{n-1}$.

\end{frame}

\begin{frame}{O modelo de linguagem com RNN}
\input{tikzfiles/LanguageModelUnfolded}
\end{frame}



\begin{frame}{GRU: Gated Recurrent Units}
\begin{center}
\includegraphics[scale=0.25]{images/gru.png}
\end{center}
\end{frame}


% \begin{frame}{LSTM: Long Short Term Memory}
% \begin{center}
% \includegraphics[scale=0.23]{images/lstm.png}
% \end{center}
% \end{frame}


\begin{frame}{Exemplo: TrumpBot\\\url{https://github.com/felipessalvatore/MyTwitterBot}}
\begin{center}
\includegraphics[scale=0.24]{images/TrumpBot.png}
\end{center}
\end{frame}


\begin{frame}{Exemplo: Funk Generator\\ \url{https://github.com/lucasmoura/funk_generator}}
\begin{quote}
\centering
É o dj que tá tocando e não sabe de nada\\ 
Eu já tô no clima e já tô no meu nome \\
Cordão de ouro no pescoço eu tô na moda \\
Com a camisa da \\
Louis \\
Vuitton \\
Pulo da morena que elas gosta\\ 
E se eu te pego no baile \\
De captiva de citroen ou de hayabusa\\ 
Tu viu a 1100 cilindradas \\
Se eu tô no litoral de cordão de ouro\\ 
De cordão de ouro no pescoço\\
\end{quote}
\end{frame}

\begin{frame}{Seq2seq: diálogo \cite{DBLP:journals/corr/VinyalsL15}}
\input{tikzfiles/seq2seq_dialog}
\end{frame}


\begin{frame}{Seq2seq: tradução \cite{luzfinger2017}}
\input{tikzfiles/Translation}
\end{frame}

\begin{frame}{Exemplo de diálogo \cite{DBLP:journals/corr/VinyalsL15}}
\begin{center}
\includegraphics[scale=0.3]{images/exemplo1.png}
\end{center}
\end{frame}


\section{Métricas}


\begin{frame}{Avaliação humana \cite{Lowe:2016}}
\begin{center}
\includegraphics[scale=0.4]{images/exemploEval1.png}
\end{center}
\end{frame}




\begin{frame}{Avaliação automática: BLEU \cite{Papineni2001}}
Essa métrica compara n-gramas (até 4) da resposta candidata com os n-gramas da refência da tradução e conta o numero de acertos. Essa métrica também penaliza traudções muito curtas:

\begin{equation}
BLUE(r, \hat{r}) = min \left(1, \frac{len(\hat{r})}{len(r)} \right) \left(\prod_{n=1}^{4} precision_{n}(r, \hat{r}) \right)^{\frac{1}{4}}
\end{equation}
em que $ precision_{n}(r, \hat{r})$ é o número de overlap de $n$ gramas de $r$ e $\hat{r}$ dividido pelo número de todos os $n$-gramas de $\hat{r}$. 

$BLUE(r, \hat{r}) \in [0,1]$

\end{frame}


\begin{frame}{Avaliação automática: problemas}
\begin{center}
\includegraphics[scale=0.23]{images/weak_corr.png}
\end{center}

"In particular, we show that these metrics (BLEU, METEOR, ROUGE) have only a small positive correlation on the chitchat oriented Twitter dataset, and no correlation at all on the technical Ubuntu Dialogue Corpus." \cite{LiuLSNCP16}

\end{frame}

\section{De diálogos abertos para pequenas tarefas}

\begin{frame}{bAbI}

\end{frame}


\section{Entailment-QA}

\begin{frame}{SICK}

\end{frame}

\begin{frame}{quora}

\end{frame}

\begin{frame}{first results}

\end{frame}




\begin{frame}[allowframebreaks]{Referências}

  \bibliography{my_references}
  \bibliographystyle{abbrv}

\end{frame}

\end{document}




\end{document}